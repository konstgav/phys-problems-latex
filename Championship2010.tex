\section{I курс}

\AddProb В U-образной трубке скользит упругая нерастяжимая нить. Скорость входящего в трубку конца равна $v$, выходящего –- $u$. 
С какой скорость движется трубка?

\AddProb Брусок массы $m$ тянут за нить так, что он движется с постоянной скоростью по горизонтальной плоскости с коэффициентом трения $\mu$. 
Найти угол $\alpha$, при котором натяжение нити минимально. Чему оно равно?

\AddProb Однородный шар радиуса $r$ скатывается без скольжения с вершины сферы радиуса $R$. Найти угловую скорость шара после отрыва от сферы. 
Начальная скорость шара пренебрежимо мала.


\section{II курс}

\AddProb На гладком столе лежит тонкое кольцо массы $M_1$ и радиуса $R$. На него сверху кладут шероховатое кольцо такого же радиуса, 
которое вращается с угловой скоростью $\omega$. Масса верхнего кольца $M_2$. Пренебрегая трением нижнего кольца о стол, определите, 
какая угловая скорость вращения колец на столе установится через большой промежуток времени. Сколько тепла выделится при установлении вращения?

\AddProb Тонкая трубка изогнута в виде почти замкнутой окружности. В этой трубке скользит упругая нерастяжимая нить. 
Скорость участка нити, входящего в трубу, равна $v$, скорость трубы равна $2v$. Обе скорости одинаково направлены. 
Какой участок нити в трубе движется с максимальной скоростью? Чему равна эта скорость?

\AddProb Металлический стержень массой $m$ и длиной $l$ подвешен на двух легких проводах длиной $R$ в магнитном поле с индукцией $B$, 
вектор которой направлен вертикально. К точкам крепления проводов подключен конденсатор емкостью $C$, заряженный до напряжения $U$. 
Сопротивление стержня и проводов пренебрежимо мало. Найти максимальный угол отклонения проводов от вертикали, 
если разрядка конденсатора происходит за очень малое время.


\section{III курс}

\AddProb Тонкая трубка изогнута в виде почти замкнутой окружности. В этой трубке скользит упругая нерастяжимая нить. 
Скорость участка нити, входящего в трубу, равна $v$, скорость трубы равна $2v$. Обе скорости одинаково направлены. 
Какой участок нити в трубе движется с максимальной скоростью? Чему равна эта скорость?

\AddProb По двум вертикальным рейкам, соединенным вверху и внизу сопротивлениями $R$, может скользить без трения проводник, 
длина которого $L$, масса $m$, сопротивление $R$. Система находится в однородном магнитном поле, индукция которого $B$ перпендикулярна плоскости рисунка. 
Найти максимальную скорость проводника в поле силы тяжести, если пренебречь сопротивлением реек.

\AddProb Теплоизолированный цилиндрический сосуд разделен теплоизолирующим поршнем на две равные части объемом $V$ каждая. 
Давление одноатомного газа в левой половине сосуда $p$, в правой $2p$, а температура одна и та же и равна $T$. Систему предоставили самой себе. 
Определите конечное давление в обеих частях сосуда после завершения переходных процессов, если поршень перемещается внутри сосуда без трения.