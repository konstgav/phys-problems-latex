\section{8 класс}

\AddProb Зачем в погребах в холодную погоду рядом с овощами ставят большие емкости с водой?

\AddProb В 100 г воды при температуре $10^{\circ}C$ опущено 40 г льда, имеющего температуру $-10^{\circ}C$. При каком соотношении воды и льда 
возникнет состояние теплового равновесия в этой системе, если она теплоизолирована? Удельная теплоемкость льда 2,5~кДж/кг.

\AddProb Два велосипедиста и пешеход одновременно отправились из пункта $A$ в пункт $B$. 
Более чем через час после выезда у первого велосипедиста сломался велосипед, и он продолжал путь пешком, двигаясь в 4,5 раза медленнее, чем на велосипеде. 
Его обгоняют: второй велосипедист –- через 5/8 ч после поломки, а пешеход –- через 10,8 ч после поломки. 
К моменту поломки второй велосипедист проехал расстояние в два раза большее, чем то, которое прошел пешеход к моменту, на 5/36 ч более позднему, 
чем момент поломки. Через сколько часов после начала движения сломался велосипед?

\AddProb Длинная вертикальная трубка погружена одним концом в сосуд с ртутью. В трубку наливают $m\,=\,0.71$~кг воды, которая не вытекает из трубки. 
Определите изменение уровня ртути в сосуде. Диаметр сосуда $D\,=\,0.06$~м, плотность ртути $\rho\,=\,13600$~кг/м$^3$. 
Толщиной стенок трубки можно пренебречь.

\AddProb Легкая цилиндрическая палочка длиной $L\,=\,20$~см и плотностью $\rho_1\,=\,800$~кг/м$^3$ погружена вертикально в жидкость плотностью 
$\rho_2\,=\,1000$~кг/м$^3$. Нижний конец палочки находится на глубине $H_0\,=\,1$~м. На какую высоту $h$ выпрыгнет палочка из жидкости, 
если ей дать возможность двигаться? Вязкостью жидкости пренебречь.



\section{9 класс}

\AddProb Найдите сопротивление «звезды» между точками $A$ и $B$, если сопротивление каждого звена равно~$R$.

\AddProb Длинная вертикальная трубка погружена одним концом в сосуд с ртутью. В трубку наливают $m\,=\,0.71$~кг воды, которая не вытекает из трубки. 
Определите изменение уровня ртути в сосуде. Диаметр сосуда $D\,=\,0.06$~м, плотность ртути $\rho\,=\,13600$~кг/м$^3$. 
Толщиной стенок трубки можно пренебречь.

\AddProb Железный шарик радусом 1 см, нагретый до $20^{\circ}C$, положен на лед. На какую глубину погрузится шарик в лед, если удельная теплоемкость железа 
$c_1\,=\,475$~Дж/кг$^{\circ}C$. Плотность льда $\rho_2\,=\,900$~кг/м$^3$, плотность железа $\rho_1\,=\,7900$~кг/м$^3$. 
Температура льда $0^{\circ}C$, удельная теплота плавления льда $\lambda\,=\,334$~кДж/кг. Теплопроводностью льда и нагревом воды пренебречь.

\AddProb Найдите скорость верхней точки пересечения двух катящихся колес (точка $A$ на рисунке) в тот момент, 
когда она находится на одной горизонтали с центром большого колеса. Скорость колес одинаковы и равны $v$, радиусы колес $r$ и~$R$.



\section{10 класс}

\AddProb Как изменится сопротивление цепи, состоящей из пяти одинаковых проводников сопротивлением R каждый, 
если добавить еще два таких же проводника, как показано штриховыми линиями на рисунке?

\AddProb На столе в один ряд лежат 10 кубиков. С какой силой нужно, взявшись за два крайних руками, сдавить кубики, чтобы оторвать их от стола? 
Массы кубиков $m$, коэффициент трения кубика о кубик~$\mu$.

\AddProb С одним молем идеального газа совершают цикл, который на PV-диаграмме изображается окружностью. Найдите максимальную температуру газа.

\AddProb На вбитом в стену гвозде на нити длиной $L$ висит маленький шарик. 
Под этим гвоздём на одной вертикали с ним на расстоянии $l < L$ вбит второй гвоздь. 
Шарик отводят вдоль стены так, что нить принимает горизонтальное положение, и отпускают без толчка. 
Найдите расстояния $l$, при которых шарик перелетит через нижний гвоздь. Нить невесома и нерастяжима, трения нет.



\section{11 класс}

\AddProb В цилиндре, закрытом подвижным поршнем, находится газ, который может просачиваться в зазор вокруг поршня. 
В опыте по изотермическому сжатию газа его объем уменьшился вдвое, а давление газа упало в 3~раза. 
Во сколько раз изменилась внутренняя энергия газа в цилиндре? (Газ считать идеальным.)

\AddProb На абсолютно гладкой проводящей поверхности удерживают два металлических кубика. 
На один кубик помещен положительный заряд~$+q$, на другой -- отрицательный~$-q$. Кубики отпускают. Через какое время они столкнутся? 
Расстояние между кубиками $h$ много меньше стороны кубика~$a$. Плотность материала~$\rho$. 

\AddProb На вбитом в стену гвозде на нити длиной $L$ висит маленький шарик. 
Под этим гвоздём на одной вертикали с ним на расстоянии $l < L$ вбит второй гвоздь. 
Шарик отводят вдоль стены так, что нить принимает горизонтальное положение, и отпускают без толчка. 
Найдите расстояния~$l$, при которых шарик перелетит через нижний гвоздь. Нить невесома и нерастяжима, трения нет.

\AddProb В широкий сосуд с жидкостью частично погружается плоский конденсатор. 
Конденсатор подключен к батарее, которая поддерживает на обкладках конденсатора постоянную разность потенциалов~$U$. 
Расстояние между пластинами~$d$, плотность жидкости~$\rho$, диэлектрическая проницаемость~$\varepsilon$. 
На какую высоту поднимется жидкость в конденсаторе? Поверхностным натяжением пренебречь.