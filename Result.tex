\documentclass[a5paper, oneside]{book}
\usepackage[T2A]{fontenc}
\usepackage[utf8]{inputenc}   %кодировка
\usepackage[russian]{babel}   %язык
\usepackage{wrapfig}  %обтекание
\usepackage{graphicx}   %графика
\graphicspath{{OldPictures/}}   %папка с картинками 
\usepackage{amsmath}
\usepackage{indentfirst} %отступ в первом абзаце
\usepackage{subcaption}   %несколько фрагментов на рисунке
\usepackage[left=1.5cm,right=1.5cm,bottom=2cm]{geometry} %настройка полей
\headsep=3mm   %расстояние между верхним колонтитулом и текстом
\newcounter{ProbCount}   %счетчик задач
\newcommand{\AddProb}{\par \medskip \addtocounter{ProbCount}{1} \textbf{Задача \arabic{ProbCount}.} }   %нумерация задач с помощью счетчика

%\oddsidemargin=0pt
%\setlength{\evensidemargin}{-10.-10mm}
%\textwidth=10cm

% Чтобы на странице с заглавием главы печаталась колонцифра + отступ в первом абзаце
%\renewcommand{\chapter}{\cleardoublepage
%\thispagestyle{empty}%
%\global\@topnum=0
%\@afterindenttrue
%\secdef\@chapter\@schapter}

\title{Сборник задач для подготовки к физическим олимпиадам}
\date{\today}

\begin{document}
\maketitle
\tableofcontents    % Оглавление

\chapter*{Предисловие}\addcontentsline{toc}{chapter}{Предисловие} 
Пособие включает в себя задания факультатива по решению задач повышенной сложности на физическом факультете Пермского государственного национального исследовательского университета. В сборник вошли задачи Краевых студенческих олимпиад Прикамья, задания Студенческих чемпионатов по физике, проводившихся в Пермском университете с 2006 года, задачи олимпиады Пермского университета по физике для школьников <<Юные таланты>>, которая проводится с 2008 года.  \\
\indent Пособие предназначено для студентов вузов, изучающих курсы общий физики <<Механика>>, <<Молекулярная физика>>, <<Электричество и магнетизм>>, а также учащихся старших классов специализированных школ.

\part{Задачи факультатива}

\chapter{Механика}
\section{Относительность движения}

%1
\AddProb Два поезда движутся навстречу друг другу со скоростью $v$ каждый. 
Определите время встречи поездов, если начальное расстояние между ними равно $L$. 
Решите задачу координатным способом, графическим способом и методом, использующим идею относительности движения.
%\medskip

\AddProb Муха летает между двумя сближающимися со скоростью $v$ стенками. 
Скорость мухи $u$. Начальное расстояние между стенками равно $L$. 
Какой путь пройдет муха до остановки, если считать, что как только она приближается к одной из стенок -- мгновенно изменяет 
направление скорости на противоположное и движется вдоль одной прямой, перпендикулярной стенкам?

\AddProb Проплывая под мостом против течения, гребец потерял соломенную шляпу. 
Обнаружив пропажу через десять минут, он повернул назад и, гребя с тем же темпом, подобрал шляпу на расстоянии 900 м ниже моста. 
Через какое время после обнаружения пропажи гребец подобрал шляпу?

\AddProb С какой скорость должен двигаться автомобиль, чтобы капли дождя не оставляли следов на заднем стекле, 
наклоненном под углом $ \alpha $? Скорость дождя $\vec v$.

\AddProb Под каким углом к направлению течения должен плыть пловец, 
чтобы переправиться на противоположный берег с наименьшим смещением из-за течения реки? 
Скорость пловца $\vec u$, скорость реки $\vec v$.

%6
\AddProb Шарик движется навстречу стенке со скорость $\vec u$, скорость движения стенки $\vec v$. 
Определите, с какой скоростью отскочит шарик от стенки после абсолютно упругого удара. 
Как изменится ответ, если стенка движется в ту же строну, что и шарик? 
Если шарик падает под углом $ \alpha $ к стенке?

\AddProb Определите кратчайшее расстояние между автомобилями, которые движутся со скоростями $v$ по перпендикулярным пересекающимся прямым. 
В начальный момент времени один автомобиль находится в центре перекрестка, а второй подъезжает к нему на расстоянии $L$. 
Как изменится ответ, если угол между прямыми равен $ \alpha $ ?

\AddProb Как изменяется расстояние между двумя каплями воды, которые свободно падают в поле силы тяжести? 
Обе капли выпущены из одной точки с интервалом времени 1 секунда.

\AddProb Два тела движутся по прямой навстречу друг другу с начальными скоростями $\vec{v}_1$ и $\vec{v}_2$ и 
постоянными ускорениями $\vec{a}_1$ и $\vec{a}_2$, направленными противоположно соответствующим скоростям в начальный момент времени. 
При каком максимальном начальном расстоянии $L_{max}$ между телами они встретятся в процессе движения?

\AddProb Открытая карусель вращается с угловой скоростью $\omega $. На карусели на расстоянии $r$ от оси вращения стоит человек. 
Идет дождь, и капли дождя падают вертикально вниз со скоростью $v_0$. 
Как человек должен держать зонт, чтобы наилучшим образом укрыться от дождя?

%11
\AddProb От колеса радиуса $R$, движущегося без проскальзывания по горизонтальной поверхности со скоростью $\vec v$, 
отрывается вертикально кусочек грязи и, пролетев по воздуху, возвращается точно в ту же точку, от которой оторвался. 
При каких условиях это возможно?

\AddProb Маленький шарик, брошенный с начальной скоростью $\vec{v}_0$ под углом $\alpha$ к горизонту, 
ударился о вертикальную стенку, движущуюся ему навстречу с горизонтально направленной скоростью $\vec v$, и отскочил в точку, из которой был брошен. 
Определите, через какое время $t$ после броска произошло столкновение шарика со стенкой? Потерями на трение пренебречь.

\begin{wrapfigure}{r}{3.5cm}
\includegraphics[width=3.5cm]{0113RelativityRings.jpg}
\includegraphics[width=3.5cm]{0114RelativityWedge.jpg}
\end{wrapfigure}

\AddProb Два колечка $O$ и $O'$ надеты на вертикальные неподвижные стержни $AB$ и $A'B'$ соответственно. 
Нерастяжимая нить закреплена в точке $A'$ и на колечке $O$ и продета через  колечко $O'$. 
Считая, что колечко $O'$ движется вниз с постоянной скоростью $\vec v$, определите скорость $\vec{v}_2$ колечка $O$, если  $\angle AOO' = \alpha$.

\AddProb На неподвижном клине, образующем угол $\alpha$  с горизонтом, лежит нерастяжимая невесомая веревка. 
Один из концов веревки прикреплен к стене в точке $A$. В точке $B$ к веревке прикреплен небольшой грузик. 
В некоторый момент времени клин начинает двигаться вправо с постоянным ускорением $\vec a$. 
Определите ускорение $\vec{a_{2}}$ грузика, пока он находится на клине.

\begin{wrapfigure}{r}{3.5cm}
\includegraphics[width=3.5cm]{012011RelativityBus.jpg}
\end{wrapfigure}

\AddProb (2011)\footnote{Здесь и далее год в скобках означает, что данная задача была предложена для решения на Краевой студенческой олимпиаде
 по физике в Пермском крае в указанном году.}По шоссе со скоростью $\vec{v}_a$ движется автобус. Человек находится на расстоянии $h$ от шоссе и на расстоянии $L$ от автобуса. 
Под каким углом $\alpha$ к шоссе со скоростью $\vec v$  должен идти человек, чтобы выйти на шоссе одновременно с автобусом? 

%16
\AddProb (2009) Самолет в безветренную погоду взлетает со скоростью $\vec v$  под углом к горизонту $\alpha_0$ . 
Внезапно начинает дуть горизонтальный встречный ветер, скорость которого $\vec u$ . 
Какой стала скорость самолета относительно земли, и какой угол составляет она с горизонтом? 

\AddProb (2010) Когда мимо пристани проплывает плот, от пристани в деревню, расположенную на расстоянии $S$ вниз по течению реки, 
отправляется моторная лодка. Она доходит до деревни за время $t$ и, сразу повернув обратно, встречает плот на расстоянии $ S_1 $ от деревни. 
Какова скорость течения реки $\vec{v}_p $?

\AddProb (2013) Самолет садится на корабль, движущийся по океану со скоростью $\vec{v}_1$ в восточном направлении. 
Скорость ветра $\vec{v}_2$ направлена на север, а самолет снижается по отношению к кораблю вертикально со скоростью $\vec{v}_3$. 
Определить величину скорости самолета по отношению к движущемуся воздуху.

\begin{wrapfigure}{r}{3.5cm}
\includegraphics[width=3.5cm]{0115RelativityPowerboats.jpg}
\end{wrapfigure}

\AddProb (2007) Два катера вышли одновременно из пунктов $A$ и $B$, находящихся на противоположных берегах реки, и двигались вдоль отрезка $AB$ длины~$l$. 
Прямая $AB$ образует угол $\alpha$ с направлением скорости течения $\vec v$. Скорости движения катеров относительно воды одинаковы. 
На каком расстоянии от пункта $B$ произошла встреча катеров, если они встретились через время $t$ после отхода от причалов?
\section{����, ��������� ��� ����� � ���������}

%1
\AddProb ������ ������ � ������ $h$ ��� ����� $\alpha$ � ���������  �� ��������� $v_0$. 
����� ���� $\beta$ ����� ���������� ��������� ����� � ���������� � ������ ������� �� �����? ���� ����� �������� ���� ��������? �� ����� ���������� $s$ �� ����������� �� ��������� ����� ������� ������ ������? 

\AddProb � ����� ����������� ��������� ����� ����������� ������ ����� ������ ������ $H$ � �������������� ������ ������� $R$?

\AddProb ������� ������� ������ � ����, �������� �� ���� �����. 
����� 1 ������� ������ ������ �� ����� � �����, ����������� �� ����� ��������� � �����. 
�� ����� ������ ��������� ���?

\AddProb ������� �������� �� ������� � ����, �������� �� ����� ������. ����� $t = 1$ � ������ �������� � ����� ����� � ��� ����. 
�� ����� ���������� $s$ �� ������� ��������� ���, ���� ��������, ��� ������� $v(0)$ � $v(t)$ ������� ���������������? 
��������� ���������� ������� $g$ = 10 �/�. ������������� ������� ������������ ����.

\AddProb (2005) ������ ������� $R$ ������� �� �������������� ������ ������ �� ��������� $v$. 
�� ����� ������������ ������ $h$ ����������� ����� ����, ������������ �� ������? 
����� ������ ���� ����������� �������� ������, ����� ��������, ��������� ������������ ������, ���������� �� �� �� ����� �����? 
��������� �� ������ $h$, ���� ������ ����� �������� � ������������?

\begin{wrapfigure}{r}{5.0cm}
\includegraphics[width=5.0cm]{0205AngleWheel.jpg}
\includegraphics[width=5.0cm]{0209AngleHouse.jpg}
\end{wrapfigure}

%6
\AddProb �� ���� �����, ������������� �� ����� ������ $h$ ��� ����� �� ���������� $l$ ���� �� �����, 
������������ ������� ��� �����: ����������� ����� �� ��������� $\vec{v}_1$ � ������������� �� ��������� $\vec{v}_2$. 
������ ����������� ���������� ����� ������� � �������� �� ��������?

\AddProb ���������� � ����������� ����� ������� ������ � ����, ������������� �� ���������� $L$ �� ����������� �� ������ $H$. 
� ����� ���������� ��������� ��� ����� �������? ������� ����������.

\AddProb (2004) ��� ����� ����������� ��������� �������� $v_0$ ����� ����������� ������ ����� ��� � ������� ������? 
��������� ����� ����� ������ $H$, ������ ����� -- ������ $h$, ������ ���� ����� $l$.

\AddProb (2010) � ����������� ����� ������� �����, ������ ��������� �� �� ������ � ���� ������, ������������� �� ����� �����������. 
���������� ������� ��� �������� ������ ����� ������� ����� $T_{1}$, � ��� �������� ������ ������ -- $T_2$. ���������� ������ �����.

%11
\AddProb �� ����� A � �, ����������� �� ����� �������������� ������, ������������ ������� ��� ����� � ����������� �� ������ ���������� $v_0$ = 20 �/�. 
���� �� ������ ������� �� �������� ����������, ������ -- �� ����������, �� ������ ����� � ����� ������ ������� �����. 
��������, ��� � ����� � ���� �������� $\alpha = 75^{\circ}$. 
����� ����� ����� $\tau$ ����� ������ ���������� ����� ������� ������ �����������? ���� ����� ��� ����������?

\AddProb (2008) � ����� ��������� �������� ��� �����: ����� 1 �������� �� ������ � ���������� ��������� $v_1$, 
� ����� 2 -- � ���������� �� ������ ��������� $v_2$, ������������ ��� ����� �� ����� 1. 
����� ���������� ����� 2 � ���������� ������� ����� 1 � 2. 
�������, ��� � ������ ������� $t$ = 0 ���������� ����� ������� $y_0$, � $\vec{v}_1$ � $\vec{v}_2$ ���������������.

\section{Мгновенный центр вращения}

\begin{ex}
Колесо радиуса $R$ катится без проскальзывания по горизонтальной поверхности со скоростью $v$. 
Найдите скорости различных точек колеса, уравнения траектории и радиус кривизны траектории в верхней точке дуги для произвольной точки на ободе колеса.
\begin{ans}
$v(\varphi) = v\sqrt{2(1- \cos \varphi)}$, $x(t) = vt - R \sin \varphi$, $y(t) = R(1 - \cos \varphi)$, $\varphi = vt / R$, $R_k = 4R$.
\end{ans}
\end{ex}

\begin{ex}
Скорость одного конца стержня равна $v$ и направлена под углом $\alpha$ к стержню. Найдите скорость другого конца, которая направлена под углом $\beta$ к стержню.
\begin{ans}
$u = v \cos \alpha / \cos \beta$
\end{ans}
\end{ex}

\begin{ex}
По гладкому горизонтальному столу свободно скользит тонкая прямая однородная палочка длины $L$. 
В некоторый момент скорость одного из концов равна $v$ и составляет прямой угол с палочкой, 
а скорость другого конца по величине равна $2v$. За какое время палочка повернется на угол $2\pi$?
\begin{ans}
$T_1 = 2 \pi L/v$, $T_2 = 2 \pi L/3v$.
\end{ans}
\end{ex}

\begin{ex}
\hspace{0pt} \\
\begin{minipage}{.65\textwidth}
(2006) Между двумя стенками, образующими прямой угол, движется по направляющим без отрыва стержень АВ длиной $l_0$. 
Скорость точки В постоянна, равна $v_0$ и направлена горизонтально. Определить скорость $v$ и ускорение $a$ точки М, расположенной на расстоянии MB = $l$ от точки В, в момент времени, когда угол между горизонтальной стенкой и стержнем АВ составляет $\alpha$.
\end{minipage}
\begin{minipage}{.35\textwidth}
\centering
\includegraphics[width=0.9\textwidth]{032006RotationKinematicsBar.jpg}
\end{minipage}
\begin{ans}
$\omega = v_0/l_0 \sin \alpha$, $R = sqrt{l^2 + l_0^2 - 2ll_0 \sin \alpha}$, $v_M = \omega R$, $a_M = \omega^2 R$.
\end{ans}
\end{ex}

\section{Бесконечно малые перемещения}

\begin{ex}
Тело движется по окружности радиуса $R$ так, что его скорость зависит от времени по линейному закону: $v = at$. Найдите зависимость ускорения тела от времени.
\begin{ans}
$a_p = \sqrt{a^2 + a^4t^4/R^2}$
\end{ans}
\end{ex}

\begin{ex}
\hspace{0pt} \\
\begin{minipage}{.65\textwidth}
Луч света падает на вращающийся экран AO, образуя на нем зайчик C. Угловая скорость вращения экрана $\omega$\,; угол, образуемый лучом света с горизонтом, равен $\alpha$. В некоторый момент времени экран занимает положение, изображенное на рисунке, при этом расстояние от оси вращения до зайчика OC = $l$. Определите, какую скорость имеет зайчик относительно экрана в указанный момент времени.
\end{minipage}
\begin{minipage}{.35\textwidth}
\centering
\includegraphics{0307InfinitesimalChangeRay.jpg}
\end{minipage}
\begin{ans}
$v = \omega l \sin \alpha$
\end{ans}
\end{ex}

\begin{ex}
\hspace{0pt} \\
\begin{minipage}{.65\textwidth}
Определить скорость точки пересечения двух лучей прожекторов, которые вращаются в противоположных направлениях с угловой скоростью $\omega$, в момент, когда угол наклона к горизонту обоих прожекторов равен $\varphi$. Расстояние между прожекторами равно $2l$.
\end{minipage}
\begin{minipage}{.35\textwidth}
\centering
\includegraphics[width = 0.9\textwidth]{0308RotationKinematicsProjectors.jpg}
\end{minipage}
\begin{ans}
$v = \omega l/ \cos^2 \varphi$
\end{ans}
\end{ex}

\begin{ex}
\hspace{0pt} \\
\begin{minipage}{.65\textwidth}
Бревно, упираясь одним концом в угол между землей и стеной, касается грузовика на высоте $h$, который отъезжает от стены со скоростью $u$. Как зависит угловая скорость вращения бревна от угла $\alpha$ между бревном и горизонтом?
\end{minipage}
\begin{minipage}{.35\textwidth}
\centering
\includegraphics[width = 0.9\textwidth]{0309RotationKinematicsLorry.jpg}
\end{minipage}
\begin{ans}
$\omega = u \sin^2 \alpha /h$
\end{ans}
\end{ex}

\begin{ex}
За лисой, бегущей прямолинейно с постоянной скоростью $v$, бежит собака таким образом, что ее скорость $u$ всегда направлена на местоположение лисы. В момент, когда векторы скоростей перпендикулярны, расстояние между ними было равно $L$. С каким ускорением при этом двигалась собака?
\begin{ans}
$a = uv/L$
\end{ans}
\end{ex}

\begin{ex} 
\hspace{0pt} \\
\begin{minipage}{.65\textwidth}
На диск радиуса $R$ намотаны две нерастяжимые нити, закрепленные в двух разных точках. При отпускании диск вращается. Когда угол между нитями у диска $\alpha$, угловая скорость вращения диска $\omega$. С какой скоростью в этот момент движется центр диска? Нити остаются натянутыми.
\end{minipage}
\begin{minipage}{.35\textwidth}
\centering
\includegraphics{0311RotationKinematicsDisc.jpg}
\end{minipage}
\begin{ans}
$\omega R/ \cos (\alpha /2)$
\end{ans}
\end{ex}

\begin{ex}
Внутри неподвижной окружности катится без скольжения другая окружность вдвое меньшего радиуса. Какую траекторию описывает при этом произвольно выбранная точка на подвижной окружности?
\begin{ans}
отрезок
\end{ans}
\end{ex}

\begin{ex}
\hspace{0pt} \\
\begin{minipage}{.65\textwidth}
Бусинка может двигаться по кольцу радиуса $R$, подталкиваемая спицей, которая вращается с угловой скоростью $\omega$ в плоскости кольца. Ось вращения спицы находится на кольце. Определить ускорение бусинки.
\end{minipage}
\begin{minipage}{.35\textwidth}
\centering
\includegraphics{0313RotationKinematicsRingAndBead.jpg}
\end{minipage}
\begin{ans}
$4 \omega^2 R$
\end{ans}
\end{ex}

\begin{ex}
По палочке, которая вращается с угловой скоростью $\omega$, ползет жук со скоростью $v$. Определите скорость и ускорение жука, когда он находится на расстоянии $L$ от оси вращения палочки.
\begin{ans}
$u = \sqrt{v^2 + \omega^2 L^2}$, $a = \sqrt{\omega^4 L^2 + 4\omega^2 v^2}$
\end{ans}
\end{ex}

\begin{ex}
(2003) Четыре черепахи находятся в вершинах квадрата со стороной $l$. Они начинают двигаться одновременно с постоянной скоростью $v$. Каждая черепаха движется по направлению к своей соседке по часовой стрелке. Где встретятся черепахи и через какое время? Найти угол между скоростью движения черепахи и одной из сторон квадрата как функцию ее координат $\varphi = \varphi (x,y)$.
\begin{ans}
В центре через $t = v/l$, $\tan \alpha = (x-y)/(x+y)$.
\end{ans}
\end{ex}
\section{Динамика}

%1
\AddProb На наклонной поверхности с углом $\alpha$ горизонту находится брусок. 
Коэффициент трения бруска о поверхность равен $\mu$. С каким ускорением будет двигаться брусок?

\begin{wrapfigure}{r}{5cm}
\includegraphics{0402DynamicsPaper.jpg}
\end{wrapfigure}

\AddProb Листы бумаги, сложенные, как показано на рисунке, склеивают свободными концами через лист таким образом, 
что получаются две самостоятельные кипы $A$ и $B$. Вес каждого листа 0.06 Н, число всех листов 200, коэффициент трения бумаги о бумагу, 
а также о стол, на котором бумага лежит, равен 0.2. Предполагая, что одна из кип удерживается неподвижно, 
определить наименьшее горизонтальное усилие $F$, необходимое для того, чтобы вытащить вторую кипу.

\begin{wrapfigure}{r}{3cm}
\includegraphics{0403DynamicsBlocks.jpg}
\end{wrapfigure}

\AddProb При какой максимальной силе $F$ верхний брусок еще не будет скользить по нижнему? 
Массы брусков $m_1$ и $m_2$, коэффициент трения между ними $\mu$, поверхность стола гладкая.

\AddProb По наклонной плоскости, образующей угол $\alpha$ с горизонтом, за веревку вытягивают ящик массы $M$. 
Коэффициент трения ящика о плоскость равен $\mu$. Под каким углом $\beta$ к горизонту следует тянуть веревку, 
чтобы равномерно двигать ящик с наименьшим усилием? Каково это усилие?

\begin{wrapfigure}{r}{2.5cm}
\includegraphics{0405DynamicsRing.jpg}
\end{wrapfigure}

\AddProb По вертикально подвешенному в поле тяжести Земли кольцу радиуса $R$ может скользить без трения шарик массы $m$. 
В начальный момент времени кольцо неподвижно, и шарик находится в нижней точке кольца. 
Как будет двигаться шарик, если кольцо начнет вращаться вокруг вертикальной оси с угловой скоростью $\omega$?

\begin{wrapfigure}{r}{3.5cm}
\includegraphics{0406DynamicsCone.jpg}
\end{wrapfigure}

%6
\AddProb К вершине прямого кругового конуса с помощью нити длиной $L$ прикреплена небольшая шайба. 
Вся система вращается вокруг оси конуса, расположенной вертикально. 
При каком числе оборотов в единицу времени шайба не будет отрываться от поверхности конуса? Угол при вершине конуса $2\alpha$.

\AddProb У края диска радиусом $R$ лежит монета. Диск раскручивается так, 
что его угловая скорость линейно растет со временем по закону $\omega = \varepsilon t$. 
В какой момент времени монета слетит с диска, если коэффициент трения между диском и монетой равен $\mu$? 
Какой угол с направлением к центру диска образует сила трения в этот момент?

\begin{wrapfigure}{r}{4cm}
\includegraphics[scale=0.5]{ForceSpring.png}
\end{wrapfigure}
\AddProb (2017) На рисунке представлен горизонтальный пружинный маятник, который может совершать колебания с частотой 2 Гц. Масса груза маятника 100 г. Горизонтальная плоскость гладкая. На маятник, находящийся в состоянии покоя в положении равновесия, начинает действовать постоянная горизонтальная сила $F = 2$ Н. 1) Определите максимальное растяжение пружины. 2) Определите максимальное растяжение пружины при условии, что сила $F$ действует только в течение времени 0,01 с.

\AddProb Найдите ускорения призмы массой $m_1$ и куба массой $m_2$, изображенных на рисунке. Трением пренебречь.

\begin{figure}[!h]
	\begin{subfigure}{.5\textwidth}
		\centering
		\includegraphics{0408DynamicsPrismAndCube.jpg}
	\end{subfigure}
	\begin{subfigure}{.5\textwidth}
		\centering
		\includegraphics{0409DynamicsWedge.jpg}
	\end{subfigure}
\end{figure}

\AddProb Клин высотой $h$ с углом наклона $\alpha$ стоит на гладкой горизонтальной поверхности. 
Масса клина $m_1$. С вершины клина начинает соскальзывать без трения брусок массой $m_2$. Найдите ускорение клина и время соскальзывания бруска.

\begin{wrapfigure}{r}{2.5cm}
\includegraphics{0410DynamicsPulley.jpg}
\end{wrapfigure}

\AddProb Найдите ускорения грузов массой $m_1$ и $m_2$ после перерезания нити. Нить и блок считать идеальными.

\begin{wrapfigure}{r}{3cm}
\includegraphics[width = 3cm]{InclinedSurfaceFriction.png}
\end{wrapfigure}
\AddProb (2016) Брусок массой 10 кг положили на наклонную плоскость с углом наклона к горизонту $\alpha = 30^{\circ}$. Коэффициент трения между бруском и плоскостью равен $\mu = 0,8$. 1) Докажите, что брусок будет покоиться относительно плоскости. 2) Определите минимальную горизонтальную силу, направленную вдоль наклонной плоскости и перпендикулярно плоскости рисунка, которую нужно приложить к бруску, чтобы его сдвинуть. 3) Определите минимальную силу‚ которую нужно приложить к бруску для того, чтобы перемещать его вверх по наклонной плоскости.

\AddProb (2015) Брусок скользит по длинной наклонной плоскости с углом наклона $\alpha = 30^{\circ}$, движущейся равномерно относительно земли по горизонтальной поверхности со скоростью $u = 10$ м/с в направлении противоположном вершине с углом $\alpha$. Начальная скорость бруска относительно плоскости равна нулю, коэффициент трения бруска о плоскость $\mu = 0,4$. 1) Определите минимальную скорость бруска относительно земли. 2) Через какое время скорость бруска относительно земли будет равна 10 м/с? 3) По какой траектории будет двигаться брусок относительно земли?
\section{����� ����}

%1
\AddProb ������ ���������� �������� ������ $l$ � ������ $m$ ������� � �������� ����� ������� �������������� ����������� ���, 
��� �� �������� ������������� � ������������ ��������� � ������� ��������� $\omega$ ������ ��� ���������������� ������� � ���������� ����� ��� �����. 
������� ��������� ������� � ����������� �� ���������� $x$ �� ��� ������.

\AddProb ������� ������ ������� �� ���� �����-����������� ������� $m_1$ � $m_2$, 
���������� ����� �������� �� �������� � �������� ������ $L$. ������� ������ �������� ������� ������.

\AddProb ��� �������� ����� $m$ � $2m$ ������� ��������� ����� ������ $l$ � �������� �� ������� �������������� ���������. 
� ��������� ������ ������� �������� ����� $2m$ ����� ����, � �������� ����� $m$ ����� $v$ � ���������� ��������������� ����. 
������� ��������� ���� � ������ �������� �������.

\AddProb (1998) �� ������� �������������� ��������� ����� ��� ���������� ������ ������ $m$ ������, 
��������� ������ �������� ���������� $k$. ������� ������ �������� �������� $v_0$ � ����������� �� ������� ������. 
������� �������� �������. ����� ����� ����� ���������� ������� ������� ��������� ������������� ��������? 

\begin{figure}[!h]
\centering
\includegraphics{051998CenterOfMassSpring.jpg}
\end{figure}

\AddProb ��� ������ $m$ �������� �� ��������� $v$ �� ���������� ��� ������ $2m$. 
������� �������� ����� ����� ����� �������� ������������ �����. 

%6
\AddProb ����������, ����� ����� ����� ������������ ������� ������ ������� ������ $m_1$ ��� ������� ������� ������������ 
� ����������� �������� ������ $m_2$. 

\AddProb ��������, ��� ��� ������� ������������� ����� ���� ���������� �����, ���� �� ������� �� ����� ��������, 
���� ������� ����� $90^{\circ}$. �������� ��� �����������.

\AddProb (2009) ��������� �������� ����� $L$ � ����� �������� �� ������ � ������� $m$ � $3m$ ��������� �� ������� �������������� ���������. 
������ ������ $m$ ����� �������� �������� $v$ � �����������, ���������������� �������. ������ ���� ��������� �������? 
��� ��������� �����, ���� �������� $v$ �������� ������ ������ $3m$? ����� ����� ������� �������� � ������������ ������� ������������� �������� � ������ � ������ �������?

\AddProb (2012) �� ��������� ������� ����� ����� ����� ������ $M$ � �������� $R$. �� ������ ��������� ���, ����� �������� $m$. 
����� ���������� ����� ��������� ��� � ����� ������ ��� �������� ���� �� ������?

\begin{wrapfigure}{r}{5.5cm}
\includegraphics{0510CenterOfMassWedge.jpg}
\end{wrapfigure}

\AddProb ���� ������ $2m$ � ����� ������� � ��������� $\alpha $ ($\cos~\alpha$~=~2/3) ��������� �� ������� �������������� ����������� �����. 
����� ����, ����������� �� ������� �����, ���������� ������ ����, ����������� ����� ������� $m$ � $3m$. 
���� ������ $3m$ ����� ��������� ����� ������������ ������������ ��, ������������ �� �����. 
���� ���� ������� ���������� ���������� �� ���������� $H$~=~27~�� �� �����, � ����� ���������. 
�� ����� ���������� ��������� ���� � ������� ������� ����� ������ $3m$ �����? ������� ����� � ������������ �� ����������.


\section{�������������� �����}

%11
\AddProb ����� ���� �������� $S$ ��������� � ������, ������������� ��������������� �����. 
�������� ���� � ����� $v$, ����� ����� ���� ������ �������� � ������� �� ������. ������ ���� �������� ���� �� ������? ��������� ���� $\rho$.

\AddProb ����������� ������� ������ $M$ �������� � �������� �������. ��� ���������� �������� ������������ ���������� ���������, 
������� ����������� ���������� ����� �� ��������� $u$ ������������ �������, ������ ������ ������� � ����� ����� $\mu$ 
(������ ������� -- ��� ����� �������, ������������� �� ������� �������). ������� ��������� �������.

\begin{wrapfigure}{r}{2.5cm}
\includegraphics{0503DistributedMassChainAndPulley.jpg}
\end{wrapfigure}

\AddProb ������� ������ ������� ���������� ����� ���� ���, ��� �� ������ ����� ������� �� ����, � ����� �����, 
����������� �������, �� ������ ������� $H$. ������� ���������, � ��� �������� � ��������. 
������� �������������� �������� �������� �������. ���� ���������, ������� ���������.

\AddProb ������ ���������� ������ ������ $m$ � �������� $R$ �������� �� ������� �������������� ����������� � 
���������� �� ������� �������� $\omega$. ������� ���� ��������� �������.

\begin{figure}[!h]
	\begin{subfigure}{.3\textwidth}
		\centering
		\includegraphics{0505DistributedMassRope.jpg}
	\end{subfigure}		
	\begin{subfigure}{.7\textwidth}
		\centering
		\includegraphics{0506DistributedMassChainAndCeiling.jpg}
	\end{subfigure}		
\end{figure}

\AddProb ������� ������ $l$ � ������ $m$ ������ �� ������� �������������� ������ �������� $R$, 
������ ������� ������� ���������� �� ������� �����, ����������� �������������� ���� $F$, � ����� ���������. 
����������: 1) �������� ���� $F$; 2) ��������� ������� � ������ ������.

%16
\AddProb ������� ������ $m$ � ������ $l$ ��������� �� ����� � �������. 
��� ���� ���������, ��� � ������ ����������� ������� �������� ���� $\alpha $ � ����������. 
������� ���������� $h$ �� ������ ����� ������� �� �������.

\begin{wrapfigure}{r}{3.5cm}
\includegraphics{0507DistributedMassChainAndTable.jpg}
\end{wrapfigure}

\AddProb (2011) ���������� ������� ������ $L$ � ������ $m$ ��������� �� ���� ���, ��� ������ ������ ��� �������� �����. 
���� ����������. ����� ����������� ���� �������� $F$ ������� �� ���� �� ����� $x$ ��� �� ������� �����. ������� ���� ������� � ���� ���������.
\section{Теорема об изменении механической энергии}

\begin{wrapfigure}{r}{4cm}
\includegraphics[width=4cm]{0601MechanicalEnergyBlocks.jpg}
\includegraphics[width=4cm]{0602MechanicalEnergyHill.jpg}
\end{wrapfigure}

%1
\AddProb На бруске длиной $l$ массой $M$, расположенном на гладкой горизонтальной поверхности, лежит маленькое тело массой $m$. 
Коэффициент трения между телом и бруском $\mu$. С какой скоростью $v$ должна двигаться система, 
чтобы после упругого удара бруска о стенку тело упало с бруска?

\AddProb Тело массой $m$ съезжает с высоты $h$ гладкой наклонной плоскости и начинает скользить по тележке массой $M$, 
находяжейся на гладкой горизонтальной поверхности. Коэффициент трения между телом и тележкой $\mu$. 
На какое расстояние переместится тело относительно тележки?

\begin{wrapfigure}{r}{4cm}
\includegraphics[width=4cm]{0603MechanicalEnergySpring.jpg}
\end{wrapfigure}

\AddProb На горизонтальной плоскости лежит тело массой $m$, соединенное с вертикальной стеной пружиной жесткостью $k$. 
В начальный момент времени пружина не деформирована. На тело начинает действовать постоянная сила $F$. 
Считая, что коэффициент трения между телом и плоскостью $\mu$ и что $F >\mu mg$, 
найдите максимальное смещение тела от начального положения и максимальную скорость тела в процессе движения.

\begin{wrapfigure}{r}{4cm}
\includegraphics[width=4cm]{0604MechanicalEnergyWedgeAndBlock.jpg}
\end{wrapfigure}

\AddProb Груз массой $m$ медленно поднимают на высоту $h$ по наклоной плоскости с помощью блока и троса. 
При этом совершается работа $A$. Затем трос отпускают, и груз скользит вниз. Найдите величину $A$, если известно, 
что скорость тела в конце спуска равна $v$.

\AddProb (2010) У основания наклонной плоскости находится брусок. Бруску сообщают некоторую начальную скорость, 
направленную вдоль плоскости вверх. На высоте $h$ скорость бруска уменьшается до значения $v_1$. 
После абсолютно упругого удара о стенку, расположенную на высоте $H > h$, брусок скользит вниз, 
и на той же высоте $h$ его скорость равна $v_2<v_1$. Определите скорость бруска в момент удара о стенку.


\section{Энергия и импульс}
%6
\AddProb На гладкой горизонтальной поверхности лежит небольшая шайба массы $m$ и гладкая горка массы $M$ высоты $H$. 
Какую минимальную скорость $v$ надо придать шайбе, чтобы она смогла преодолеть барьер?

\AddProb (2012) Две лодки идут параллельными курсами навстречу друг другу с одинаковыми скоростями $v$. 
Когда лодки встречаются, с одной лодки на другую перебрасывают груз массой $m$, а затем со второй лодки на первую перебрасывают такой же груз. 
В другой раз грузы перебрасывают из лодки в лодку одновременно. В каком случае скорости лодок после перебрасывания грузов будут больше? 
Масса каждой лодки $M$.

\AddProb (2006) Лягушка массы $m$ сидит на конце доски массы $M$ и длины $L$. Доска плавает по поверхности пруда. 
Лягушка прыгает под углом $\alpha$ к горизонту вдоль доски. Какой должна быть начальная скорость лягушки, 
чтобы она оказалась после прыжка на противоположном конце доски? Как изменится ответ, если 1) доска и лягушка сносятся течением со скоростью $u$,
 и лягушка прыгает по направлению против течения; 2) доска испытывает при своем движении постоянную силу сопротивления воды $F$?

\AddProb (2001) Трактор массы $m$ рывками перемещает груз массы $M>m$. Они соединены прочным нерастяжимым тросом длины $L$. 
В начальный момент трактор находится рядом с грузом. Сколько рывков надо сделать трактору, чтобы переместить его на расстояние $s$? 
Считать, что коэффициент трения трактора и груза о землю одинаков.

\AddProb Два груза массы $m$, соединенные пружиной жесткостью $k$, находятся на гладком горизонтальном столе. 
Одному из тел сообщают скорость $v$ и измеряют максимальное растяжение пружины. 
В ходе опыта пружина лопнула при растяжении, равном половине максимального. С какими скоростями грузы поедут по столу?

\begin{wrapfigure}{r}{4cm}
\includegraphics[width=4cm]{0611EnergyAndImpulseBar.jpg}
\end{wrapfigure}

%11
\AddProb Два тела малых размеров массой $m$ каждое соединены стержнем пренебрежимо малой массы длиной $l$. 
Система из начального положения у вертикальной гладкой стены приходит в движение. 
Нижнее тело скользит без трения по горизонтальной поверхности, верхнее - по вертикальной. 
Найдите значение скорости нижнего тела, при котором верхнее оторвется от вертикальной стенки.

\begin{wrapfigure}{r}{2.5cm}
\includegraphics{0612EnergyAndImpulseBalls.jpg}
\end{wrapfigure}

\AddProb (2002) Три одинаковых шарика массы $m$ каждый, скрепленные двумя невесомыми стержнями длиной $l$, 
поставили на гладкую горизонтальную плоскость. Найти скорость верхнего шарика в момент удара о плоскость.

\AddProb (2018) Студент стреляет из рогатки шариком массой 20 г, доведя усилие при растяжении резинки до 50 H. При этом длина резинки увеличивается в три раза. Резника рогатки имеет общую длину в нерастянутом состоянии 30 см, сложена вдвое. 1) Определите скорость шарика пренебрегая массой резинки. 2) Определите скорость шарика, если масса резинки 50 г.

\AddProb (2018) Автомобиль движется с постоянной скоростью по горизонтальному шоссе. Мощность, развиваемая двигателем
автомобиля, равна 60 кВт, эффективная площадь сопротивления автомобиля 0,7 м\textsuperscript{2} (площадка соударений молекул воздуха с автомобилем, перпендикулярная скорости автомобиля). КПД бензинового двигателя 25\%, удельная теплота сгорания бензина $46 \cdot 10^6$ Дж/кг, плотность бензина 710 кг/м\textsuperscript{3}. Температура окружающего воздуха $20^{\circ}$ С, атмосферное давление 10\textsuperscript{5} Пa, молярная масса воздуха 29 г/моль. 1) Сколько литров бензина
тратит автомобиль за 1 час? 2) C какой скоростью движется автомобиль?

\begin{wrapfigure}{r}{2.5cm}
\includegraphics[width = 2.5cm]{PlatformWithOscillator.png}
\end{wrapfigure}
\AddProb (2016) Платформа массой $М$ стоит на гладкой горизонтальной плоскости. На платформе закреплен штатив, к которому на нити длиной $l$ подвешен груз массы $m$. Грузу сообщают горизонтальную скорость $v_0$, при этом максимальный угол отклонения нити от вертикали не превышает $90^{\circ}$. l) Определите максимальную высоту подъема груза. 2) Определите максимальную скорость платформы при качаниях груза. 3) Определите силу натяжения нити и момент времени, когда скорость платформы максимальна.
\section{Вращательное движение твердого тела}
%TODO Ижевские олимпиады - вращательное движение

\begin{ex}
Определить ускорение тел и натяжение нити на машине Атвуда, предполагая, что $m_2>m_1$. Момент инерции блока относительно геометрический оси равен $I$, радиус блока $r$. Массу нити считать пренебрежимо малой. 
\begin{ans}
$a=g(m_1-m_2)/(m_1 +m_2 + I/r^2)$
\end{ans}
\end{ex}

\begin{ex}
\hspace{0pt} \\
\begin{minipage}{.65\textwidth}
(2010) На горизонтальной шероховатой поверхности лежит катушка ниток массой $m$. Ее момент инерции относительно собственной оси $I$, внешний радиус $R$, радиус намотанного слоя ниток $r$. Катушку без скольжения начали тянуть с постоянной силой $F$, направленной под углом $\alpha$ к горизонту. Найти ускорение центра катушки.
\end{minipage}
\begin{minipage}{.35\textwidth}
\centering
\includestandalone{Pictures/0717RotationDynamicsReelOfThread}
\end{minipage}
\begin{ans}
$a = F(\cos \alpha - r/R)/(m+I/R^2)$
\end{ans}
\end{ex}

%Сивухин-369
\begin{ex}
(2015) Гимнаст массы 80 кг, крутя солнышко на турнике. остановился и сделал стойку на руках (вверх ногами). Затем. немного отклонившись. начал вращаться, удерживая тело a прямом положении. Оценить максимальную силу натяжения, возникающую в каждой руке гимнаста.
\begin{ans}
$T=2mg$
\end{ans}
\end{ex}

%Сивухин-353
\begin{ex}
Монета массы $m$ и радиуса $r$, вращаясь в горизонтальной плоскости вокруг своей геометрической оси с угловой скоростью $\omega$, вертикально падает на горизонтальный диск и прилипает к нему. В результате диск приходит во вращение вокруг своей оси. Возникающий при этом момент сил трения в оси диска постоянен и равен $M_0$. Через какое время вращение диска прекратится? Сколько оборотов сделает диск до полной остановки? Момент инерции диска относительно его геометрической оси~$I_0$. Расстояние между осями диска и монеты равно~$d$.
\begin{ans}
$t=mr^2\omega/2M_0$, $N=M_0t^2/2I$, $I = I_0+m(d^2+r^2/2)$
\end{ans}
\end{ex}

\begin{ex}
Каким участком сабли следует рубить лозу, чтобы рука не чувствовала удар? Саблю считать однородной пластиной.
\begin{ans}
$l=2L/3$
\end{ans}
\end{ex}

%Сивухин-344
\begin{ex}
Сплошной однородный короткий цилиндр радиуса $R$, вращающийся вокруг своей геометрической оси со скоростью $\nu$ об/с, ставят в вертикальном положении на горизонтальную поверхность. Сколько оборотов $N$ сделает цилиндр, прежде чем вращение его полностью прекратится? Коэффициент трения скольжения между основанием цилиндра и поверхностью, на которую он поставлен, не зависит от скорости вращения и равен~$\mu$.
\begin{ans}
$N=3 \pi R \nu^2/(4\mu g)$
\end{ans}
\end{ex}

%Иродов-1.290
\begin{ex}
Однородный тонкий негнущийся стержень массой $m$ поддерживается в горизонтальном положении двумя вертикальными опорами у концов стержня. В начальный момент времени $t~=~0$ одна из опор выбивается. Найти силу, которая действует на вторую опору сразу же после этого момента.
\begin{center}
\includestandalone{Pictures/0705RotationDynamicsHorzontalBar}
\end{center}
\begin{ans}
$N = mg/4$
\end{ans}
\end{ex}

%Ижевск
\begin{ex}
\hspace{0pt} \\
\begin{minipage}{.65\textwidth}
Тонкий стержень массой $M$ и длиной $L$ свободно падает в вертикальной плоскости из начального положения, в котором угол между стержнем и горизонтальной плоскостью составлял $\alpha = 30^{\circ}$. Определите давление стержня на плоскость в момент удара, считая точку опоры стержня о плоскость неподвижной.
\end{minipage}
\begin{minipage}{.35\textwidth}
\centering
\includestandalone{Pictures/0709RotationDynamicsFallOfBar}
\end{minipage}
\begin{ans}
$N=\sqrt{10}Mg/4$
\end{ans}
\end{ex}

%Иродов-1.331
\begin{ex}
Сплошной цилиндр без проскальзывания катится со скоростью $v$ по горизонтальной плоскости, которая переходит в наклонную поверхность с углом $\alpha$. Радиус цилиндра $R$. Найти максимальное значение скорости цилиндра, при которой он перейдет на наклонную плоскость без скачка. Скольжения нет.
\begin{ans}
$v=\sqrt{gR(7\cos \alpha - 4)/3}$
\end{ans}
\end{ex}

\begin{ex}
(2003) Обруч радиуса $R$ бросают вперед со скоростью $v_0$ и сообщают ему одновременно угловую скорость $\omega_0$. 
Определить минимальное значение угловой скорости  $\omega_{0 \min}$, при котором обруч после движения с проскальзыванием покатится назад. 
Найти значение конечной скорости $v$, если $\omega_0 > \omega_{0 \min} $. Трением качения можно пренебречь.
\begin{ans}
$\omega_{0 \min} = v_0/R$, $v= (\omega_0R - v_0)/2$
\end{ans}
\end{ex}

%Сивухин-391
\begin{ex}
(2008) Сплошной однородный цилиндр, ось которого горизонтальна, движется без вращения по гладкой горизонтальной плоскости в направлении, перпендикулярном к его оси. В некоторый момент цилиндр достигает границы, где поверхность становится шероховатой и возникает постоянная (не зависящая от скорости) сила трения скольжения, а трение качения отсутствует. Каково будет движение цилиндра после перехода границы? Как распределится кинетическая энергия поступательного движения цилиндра?
\begin{sol}
Задача 391 сборника задач Сивухина.
\end{sol}
\begin{ans}
Движение после перехода границы будет сначала равнозамедленное, затем с постоянной скоростью; $1/3$ энергии превратится в тепло, $2/9$ во вращательную энергию и $4/9$ останется в виде поступательной энергии движения.
\end{ans}
\end{ex}

%----Закон сохранения момента импульса------
\begin{ex}
(2001) Пуля массы $m$, летящая горизонтально со скоростью $v$, попадает в покоящийся на шероховатой горизонтальной поверхности деревянный шар массой $M \gg m$  и радиусом $R$ на расстоянии $l$ ниже центра масс и застревает в нем. Найти установившуюся скорость шара.
\begin{ans}
$u = \frac{5mv}{7M}\left( 1 - l/R \right)$
\end{ans}
\end{ex}

%Иродов-1.310
\begin{ex}
(2004) Двум дискам радиусами $R_1$ и $R_2$ сообщили одну и ту же угловую скорость $\omega_0$, а затем их привели в соприкосновение, и система через некоторое время пришла в новое установившееся состояние движения. Оси дисков неподвижны, трения в осях нет. Моменты инерции относительно их осей вращения равны $I_1$ и $I_2$. Найти приращение момента импульса системы и приращение ее механической энергии.
\begin{ans}
$\Delta L = -4I_1I_2\omega_0/(I_1+I_2)$, $\Delta E = -2I_1I_2\omega_0^2/(I_1+I_2)$
\end{ans}
\end{ex}

%Сивухин-353
\begin{ex}
\hspace{0pt} \\
\begin{minipage}{.65\textwidth}
Тонкий стержень массы $m$ и длины $L$ подвешен за один конец и может вращаться без трения вокруг горизонтальной оси. К той же оси подвешен на нити дины $l$ шарик такой же массы $m$. Шарик отклоняют на некоторый угол и отпускают. При какой длине нити шарик после удара о стержень остановится? Удар абсолютно упругий.
\end{minipage}
\begin{minipage}{.35\textwidth}
\centering
\includestandalone{Pictures/0704RotationDynamicsBarAndBall}
\end{minipage}
\begin{ans}
$l = L/\sqrt{3}$
\end{ans}
\end{ex}

\begin{ex}
(2002) Шарик массой $m$ подвешен на нерастяжимой нити длиной $l$ и отклонен на малый угол от положения равновесия. В той же точке, что и нить, подвешен стержень длиной $1.5l$. Какова должна быть масса стержня $M$, чтобы в результате столкновения шарик остановился? Удар абсолютно упругий. Определить период колебаний шарика.
\begin{ans}
$M=2l/\sqrt{3}$, $T = 2\pi \sqrt{l/g}$
\end{ans}
\end{ex}

%Сивухин-408
\begin{ex}
(2007) На гладком горизонтальном столе лежит однородный твердый стержень длины $l$ и массы $M$, в край которого ударяет твердый шарик массы $m$, движущийся со скоростью $v_0$, перпендикулярной к оси стержня. Считая удар идеально упругим и предполагая, что силы трения между поверхностью стола и лежащими на ней телами пренебрежимо малы, вычислить угловую скорость вращения стержня после удара.
\begin{sol}
Решение задачи 408 сборника задач Сивухина.
\end{sol}
\begin{ans}
$\omega = \frac{12mv_0}{(4m+M)l}$
\end{ans}
\end{ex}

\begin{ex}
Шарик массой $m$ летит со скоростью $u_0$ навстречу покоящемуся стержню массой $M = 2m$ и длиной $2L$. Направление движения шарика перпендикулярно стержню и удалено на расстояние $l$ от его центра. После удара скорость шарика становится равной $u_1$, а стержня -- $V$. При этом стержень начинает вращаться с угловой скоростью $\omega$. Требуется определить 1) при каком $l$ шарик после удара остановится, а также 2) скорость шарика, стержня и угловую скорость вращения стержня, если шарик ударяет в конец стержня.
\begin{ans}
1) $l = L/\sqrt{3}$; 2) $u_1 = u_0/3$, $V=u_0/3$, $\omega = u_0/l$.
\end{ans}
\end{ex}

%Сивухин-397, Черепанов
\begin{ex}
Как надо ударить кием по бильярдному шару, чтобы при столкновении с другим (неподвижным) шаром 1) оба шара стали двигаться вперед (удар с накатом), 2) первый шар остановился, а второй двигался вперед (удар с остановкой), 3) второй шар двигался вперед, а первый откатился назад (удар с оттяжкой)? Предполагается, что удар наносится горизонтально в вертикальной плоскости, проходящей через центр шара и точку касания его с плоскостью бильярдного стола.
\begin{ans}
1) верхний удар $x > 2R/5$; 2) нормальный удар $x = 2R/5$; 3) нижний удар $x < 2R/5$.
\end{ans}
\end{ex}

%Черепанов - задача про игру в городки
\section{Решение дифференциальных уравнений}

\begin{ex}
Экспериментально установлено, что при движении пули массы $m$ в деревянной доске сила сопротивления пропорциональна скорости пули по закону $\vec{F} = - \alpha \vec{v}$. Какой путь пройдет пуля в доске до остановки, если начальная скорость пули $v_0$?
\begin{ans}
$x(t) = \frac{mv_0}{\alpha}\left( 1 - e^{-\alpha t /m}\right)$, $s = mv_0/\alpha$
\end{ans}
\end{ex}

\begin{ex}
При движении тел в воздухе на них действует сила сопротивления пропорциональная квадрату скорости $F = - \alpha v^2$. По какому закону изменяются скорость и пройденный путь телом массы $m$ при движении без начальной скорости?
\begin{ans}
$v(t) = \sqrt{\frac{mg}{\alpha}} \tanh (kt/m)$, $s = \frac{m}{\alpha} \sqrt{\frac{mg}{\alpha}} \cosh^2 (kt/m)$
\end{ans}
\end{ex}

%Черепанов
\begin{ex}
Стальной шарик падает с высоты $h$ с нулевой начальной скоростью на стальную плиту. Сопротивление воздуха пропорционально квадрату скорости шарика, коэффициент пропорциональности $k$. Удар о плиту абсолютно упругий. На какую высоту $\Delta h$ шарик не долетит до начального положения при первом отскоке?
\begin{ans}
$\Delta h = \frac{1}{2\beta} \log \frac{e^{2\beta h}}{2 - e^{-2\beta h}}$, $\beta = k/m$
\end{ans}
\end{ex}

%www.math24.ru
\begin{ex}
В начальный момент цепочка длиной $L$ свисает над краем стола таким образом, что сила тяжести уравновешена силой трения. В результате небольшого смещения $\varepsilon$ цепочка начинает скользить. Определить время $T$, за которое цепочка полностью соскользнет со стола. Коэффициент трения между цепочкой и поверхностью стола равен $\mu$.
\begin{ans}
$T = \sqrt{\frac{L}{(1+\mu) g}} \arccosh \frac{L}{\varepsilon (1+\mu)g} = \sqrt{\frac{L}{(1+\mu) g}} \log \left[ \frac{L}{\varepsilon (1+\mu) g} - \sqrt{\frac{L^2}{\varepsilon^2 (1+\mu)^2 g^2} - 1}\right]$
\end{ans}
\end{ex}

\begin{ex}
\hspace{0pt} \\
\begin{minipage}{.65\textwidth}
Бачок имеет форму параллелепипеда с площадью основания $S^*$. При его наполнении водой поднимается поплавок, который постепенно закрывает кран подачи воды. Для простоты будем считать, что с увеличением уровня воды $h$ в бачке площадь отверстия крана уменьшается по линейному закону $S = S_0 (1 - h/h_{\max})$. Скорость подачи воды постоянна и равна $v$. За какое время бачок полностью наполнится водой?
\end{minipage}
\begin{minipage}{.35\textwidth}
\centering
\includegraphics[width = 0.9 \textwidth]{ToileteTank.png}
\end{minipage}
\begin{ans}
$h = h_{\max}(1-e^{-S_0vt/S^*h_{\max}})$
\end{ans}
\end{ex}

\begin{ex}
В цилиндрическом баке с площадью основания $S$ находится вода, уровень которой расположен на высоте $h_0$. Вблизи дна бака имеется небольшое отверстие площадью $s$. Как изменяется со временем уровень воды баке при истечении из отверстия?
\begin{ans}
$h(t) = \left(\sqrt{h_0} - s t \sqrt{g/2} /S \right)^2$
\end{ans}
\end{ex}

\begin{ex}
Вывести дифференциальное уравнение вытекания жидкости из конического сосуда и определить полное время вытекания $T$. Радиус верхнего основания конического сосуда равен $R$, а радиус нижнего основания (отверстия) $a$. Начальная уровень жидкости составляет $H$.
\begin{sol}
Изменение уровня жидкости на высоте \(z\) описывается дифференциальным уравнением
\[S\left( z \right)\frac{{dz}}{{dt}} = q\left( z \right),\]
где \(S\left( z \right)\) -- площадь поперечного сечения сосуда на высоте \(z,\) а \(q\left( z \right)\) -- поток жидкости, зависящий от высоты $z$.

Принимая во внимание геометрию сосуда, можно предположить, что \[q\left( z \right) =  - \pi {a^2}\sqrt {2gz} ,\] где \(a\) -- радиус отверстия на дне конического сосуда. Учитывая, что отверстие достаточно малое, осевое сечение можно рассматривать как треугольник. Из подобия треугольников следует, что \[\frac{R}{H} = \frac{r}{z}.\] Следовательно, площадь поверхности жидкости на высоте \(z\) будет равна
\[
{S\left( z \right) = \pi {r^2} }
= {\pi {\left( {\frac{{Rz}}{H}} \right)^2} }
= {\frac{{\pi {R^2}{z^2}}}{{{H^2}}}.}
\]
Подставляя \(S\left( z \right)\) и \(q\left( z \right)\) в дифференциальное уравнение, имеем:
\[\frac{{\pi {R^2}{z^2}}}{{{H^2}}}\frac{{dz}}{{dt}} =  - \pi {a^2}\sqrt {2gz} .\]
После простых преобразований получаем следующее дифференциальное уравнение:
\[{z^{\large\frac{3}{2}\normalsize}}dz =  - \frac{{{a^2}{H^2}}}{{{R^2}}}\sqrt {2g} dt.\]
Проинтегрируем обе части, учитывая, что уровень жидкости уменьшается от начального значения \(H\) до нуля за время \(T:\)
\[
{\int\limits_H^0 {{z^{\large\frac{3}{2}\normalsize}}dz}  =  - \int\limits_0^T {\frac{{{a^2}{H^2}}}{{{R^2}}}\sqrt {2g} dt} ,}\;\; 
{\Rightarrow \left. {\left( {\frac{{{z^{\large\frac{5}{2}\normalsize}}}}{{\frac{5}{2}}}} \right)} \right|_0^H = \frac{{{a^2}{H^2}}}{{{R^2}}}\sqrt {2g} \left[ {\left. {\left( t \right)} \right|_0^T} \right],}\;\; 
{\Rightarrow \frac{2}{5}{H^{\large\frac{5}{2}\normalsize}} = \frac{{{a^2}{H^2}}}{{{R^2}}}\sqrt {2g} T,}\;\; 
{\Rightarrow \frac{1}{5}\sqrt {\frac{{2H}}{g}}  = \frac{{{a^2}}}{{{R^2}}}T,}\;\; 
{\Rightarrow T = \frac{{{R^2}}}{{5{a^2}}}\sqrt {\frac{{2H}}{g}} .}
\]
\end{sol}
\begin{ans}
$z^{\frac{3}{2}}zt = - \frac{a^2H^2}{R^2}\sqrt{2g}dt$, $T = \frac{R^2}{5a^2}\sqrt{\frac{2H}{g}}$
\end{ans}
\end{ex}

%Юмашев
\begin{ex}
\hspace{0pt} \\
\begin{minipage}{.65\textwidth}
По длинному хорошо растяжимому жгуту, один конец которого прикреплен к стене, а другой оттягивается с постоянной скоростью $u$, ползет муравей. Скорость муравья относительно жгута постоянна и равна $v$ и направлена в сторону движущегося конца жгута. Доберется ли муравей до конца жгута? За какое время? Начальная длина жгута $L_0$, муравей стартует от неподвижного конца.
\end{minipage}
\begin{minipage}{.35\textwidth}
\centering
\includegraphics[width = 0.9 \textwidth]{AntAndTow.png}
\end{minipage}
\begin{sol}
см. Юмашев Интегралы и производные в физике
\end{sol}
\begin{ans}
$\tau = L_0(e^{u/v} - 1)/u$
\end{ans}
\end{ex}

%Иродов1.105
\begin{ex}
\hspace{0pt} \\
\begin{minipage}{.65\textwidth}
Небольшую шайбу $A$ положили на наклонную плоскость, составляющую угол $\alpha$ с горизонтом и сообщили начальную скорость $v_0$. Найти зависимость скорости шайбы от угла $\varphi$, если коэффициент трения $k = \tan \alpha$ и в начальный момент $\varphi_0 = \pi/2$.
\end{minipage}
\begin{minipage}{.35\textwidth}
\centering
\includegraphics[width = 0.9 \textwidth]{WasherOnSurface.png}
\end{minipage}
\begin{ans}
$v=v_0/(1+\cos \varphi)$
\end{ans}
\end{ex}

%Черепанов
\begin{ex}
Наклонная плоскость имеет угол $\alpha$ с горизонтом. Тело, лежащее на наклонной плоскости, толкнули в горизонтальном направлении с начальной скоростью $v_0$. Коэффициент трения тела о плоскость равен $\mu = k \tan \alpha (k > 1)$. Через какое время тело остановится и какой путь пройдет до остановки?
\begin{ans}
$\tau = \frac{kv_0}{g\sin\alpha(k^2-1)}$, $s = \frac{2v_0^2k^2}{4kg\sin\alpha(k^2-1)}$
\end{ans}
\end{ex}

%Туймадаа
\begin{ex}
С вертикальной скалы высотой $H$ брошен горизонтально со скоростью $v_0$ камень массой $m$. Спустя некоторое время он стал двигаться с постоянной скоростью. Считая, что сила сопротивления воздуха пропорциональна скорости, найти расстояние по горизонтали $L$, на которое камень удалится от скалы в момент падения, и время движения $t$.
\begin{ans}
$L = mv_0/k$, $\tau = kH/mg + m/k$
\end{ans}
\end{ex}

%Черепанов
\begin{ex}
(2007) Бусинка находится в наинизшей точке вертикально расположенной неподвижной шероховатой окружности радиуса $R$. Какую минимальную скорость надо сообщить бусинке, чтобы она достигла горизонтального диаметра окружности? Коэффициент трения равен $\mu$.
\begin{ans}
$v_0^2 = \frac{2gR}{4\mu^2+1}\left( 3\mu e^{\mu \pi} - 2\mu^2 +1\right)$
\end{ans}
\end{ex}

%Черепанов
\begin{ex}
(2008) Сферическая капля воды движется в однородном поле тяжести в среде, в которой за счет конденсации происходит увеличение массы капли, пропорциональное ее поверхности с коэффициентом пропорциональности $\alpha$. Найти скорость капли в зависимости от времени, если в начальный момент времени капля была неподвижна, ее масса равнялась $m_0$. Плотность воды $\rho$.
\begin{ans}
$v=gt/4+g\gamma t (\beta^2t^2+3\beta t \gamma+3\gamma^2)/4(\beta t + \gamma)$, $\beta = 4\pi \alpha(3m/4\pi\rho)^{2/3}/3$, $\gamma = m_0^{1/3}$
\end{ans}
\end{ex}

%Жухарев
\begin{ex} (2008) В данной плоскости движутся две точки: точка 1 движется по прямой с постоянной скоростью $v_1$, а точка 2 – с постоянной по модулю скоростью $v_2$, направленной все время на точку 1. Найти траекторию точки 2 и координату места встречи 1 и 2. Считать, что в начальный момент времени расстояние между точками $y_0$ и $\vec{v_2} \bot \vec{v_1}$.
\begin{ans}
$x=\frac{h}{2(1+u/v)}(\frac{y}{h})^{1+u/v} - \frac{h}{2(1-u/v)}(\frac{y}{h})^{1-u/v} - \frac{h}{2(1+u/v)}+\frac{h}{2(1-u/v)}$, $x_0=h\frac{u}{v}\left(1-\frac{u^2}{v^2}\right)^{-1}$
\end{ans}
\end{ex}

\chapter{Молекулярная физика}
\section{Газовые законы}

\begin{wrapfigure}{r}{3cm}
\includegraphics{082001GasLawsProcess.jpg}
\end{wrapfigure}

%1
\AddProb (2001) Какой максимальной температуры достигнет газ в процессе, изображенном на рисунке? Показатель адиабаты $\gamma$ считать известным.

\AddProb (2010) Посередине откаченной и запаянной с обоих концов горизонтально расположенной трубки длины $L$ находится столбик ртути длины $h$. 
Если трубку поставить вертикально, столбик ртути смеситься на расстояние $x$. Какое первоначальное давление в трубке? Плотность ртути $\rho$.

\AddProb (2013) В баллон, вместимостью $V$, при давлении $p$ нагнетают воздух. 
За какое время $t$ он будет накачан до давления $p_n$, если компрессор за время $\tau$ засасывает объем $V_0$ атмосферного воздуха? 
Температуру считать неизменной, а атмосферное давление равным $p_0$. Как изменится ответ, если воздух откачивать из баллона?

\begin{wrapfigure}{r}{3.5cm}
\includegraphics[scale=0.5]{082012GasLawsGlass.jpg}
\end{wrapfigure}

\AddProb (2012) На поверхности жидкости плотностью $\rho$ плавает тонкостенный цилиндрический стакан высотой $h$, наполовину погруженный в жидкость. 
На какую глубину $h_1$ погрузится стакан в жидкость, если его осторожно положить на поверхность жидкости вверх дном? 
На какую глубину $h_2$ нужно утопить перевернутый вверх дном стакан, чтобы он вместе с заключенным в нем воздухом пошел ко дну? Давление атмосферы $p_0$.

\begin{wrapfigure}{r}{2.5cm}
\includegraphics[scale=0.25]{0805GasLawsPiston.jpg}
\end{wrapfigure}

\AddProb В вертикальном закрытом сосуде имеется поршень, который может перемещаться без трения. 
По обе стороны от поршня находятся одинаковые массы одного и того же газа. 
При температуре $T$ объем верхней части в $n$ раз больше, чем объем нижней. 
Каким будет соотношение этих объемов, если повысить температуру до значения $T_2$?


\section{Молекулярно-кинетическая теория}

\begin{wrapfigure}{r}{5cm}
\includegraphics[scale=0.5]{0806KineticTheoryTwoVessels.jpg}
\end{wrapfigure}

%6
\AddProb Два сосуда одинакового объема соединены трубками. Диаметр одной из трубок велик, 
а другой мал по сравнению со средней длиной свободного пробега молекул газа, находящегося в сосуде. 
Первый сосуд поддерживается при температуре $T$, а второй при температуре $4T$. 
В каком направлении будет перетекать газ по узкой трубке, если перекрыть широкую трубку? 
Какая масса газа перейдет при этом из одного сосуд в другой, если общая масса газа в обоих сосудах равна $M$?

\begin{wrapfigure}{r}{3.2cm}
\includegraphics[scale=0.25]{0807KineticTheoryHelium.jpg}
\end{wrapfigure}

\AddProb Теплоизолированная полость небольшими малыми одинаковыми отверстиями соединена с двумя объемами, содержащими газообразный гелий. 
Давления в этих объемах поддерживаются одинаковыми и равными $P$, а температуры поддерживаются равными в одном из объемов $T$, в другом $2T$. 
Найдите установившиеся давление и температуру внутри полости.

\AddProb Плоская поверхность нагрета неравномерно, так что вдоль нее поддерживается градиент температуры $dT/dx$. 
В этих условиях газ, примыкающий к поверхности, приходит в движение вдоль поверхности. Это явление называют тепловым скольжением. 
Объясните его механизм и оцените скорость теплового скольжения. Необходимые параметры считать известными.

\AddProb (2008) Оценить по порядку величины установившуюся скорость, с которой будет двигаться в сильно разреженном воздухе плоский диск, 
одна из сторон которого нагрета до температуры $T_1$, а другая до температуры $T_2$, $T_1>T_2$. Температура воздуха равна $T$.

\AddProb (2009) Каково должно быть максимальное значение температурного градиента $dT/dz$ атмосферного воздуха, 
чтобы он мог находиться в устойчивом механическом равновесии? Воздух считать двухатомным газом с относительной молекулярной массой $\mu$. 
Ускорение свободного падения $g$ не зависит от высоты над поверхностью земли.
\section{Закон сохранения энергии}

\begin{wrapfigure}{r}{2.5cm}
\includegraphics[scale=1]{0901LawOfConservationOfEnergyTwoPistons.jpg}
\end{wrapfigure}

%1
\AddProb В длинной теплоизолированной трубке между одинаковыми поршнями массы $m$ находится 1 моль одноатомного идеального газа при температуре $T_0$. 
В начальный момент скорости поршней направлены в одну сторону и равны $v$ и $3v$. До какой максимальной температуры нагреется газ? 
Поршни тепло не проводят, массой газа по сравнению с массой поршней пренебречь.

\AddProb (2004) В длинной горизонтальной трубе могут скользить без трения два поршня, массы которых $m$ и $2m$. 
Между ними находится некоторое количество одноатомного газа при давлении $p$ и объеме $V$. 
В этот момент легкий поршень движется к тяжелому со скоростью $v_0$, тяжелый поршень покоится. 
Оцените максимальную скорость тяжелого поршня.

\AddProb (2003) Внутри закрытого теплоизолированного цилиндра с идеальным газом находится легкоподвижный теплопроводящий поршень. 
При равновесии поршень делит цилиндр на две равные части и температура газа равна $T_0$. Поршень начали медленно перемещать. 
Найти температуру газа как функцию отношения $\eta$ объема большей части к объему меньшей части. Показатель адиабаты газа~$\gamma$.

\begin{wrapfigure}{r}{3cm}
\includegraphics[scale=0.25]{0904LawOfConservationOfEnergyEfficiency.jpg}
\end{wrapfigure}

\AddProb Найдите КПД тепловой машины, цикл которой состоит из двух изохор и двух изобар, а рабочим телом является идеальный одноатомный газ. 
Середины нижней изобары и левой изохоры лежат на изотерме, соответствующей температуре $T_1$, 
а середины верхней изобары и правой изохоры -- на изотерме, соответствующей температуре~$T_2$.

\AddProb (2014) В вертикальном цилиндрическом сосуде с теплонепроницаемыми стенками под поршнем массы $m$~=~100~г находится 5 моль неона 
(молярная масса 20 г/моль). В начальный момент поршень закреплен. После того, как поршень освободили, объем газа увеличился в 2 раза. 
Определите конечную температуру газа, если его начальная температура равна $T_0$ = 300~К. Считайте, что над поршнем вакуум. 
Трением между поршнем и стенками сосуда отсутствует.

\begin{wrapfigure}{r}{3cm}
\includegraphics[width=3cm]{092002LawOfConservationOfEnergyProcess.jpg}
\end{wrapfigure}

%6
\AddProb (2002) Чему равна работа 1 моля идеального газа в круговом процессе, показанном на рисунке? Температуры $T_1$ и $T_2$ известны.

\section{Теплоемкость}

\AddProb (1998) Имеется идеальный газ, теплоемкость которого при постоянном объеме равна $C_V$. 
Найдите молярную теплоемкость этого газа как функцию объема, если давление газа меняется по закону $p=p_0~e^{\alpha V}$ ($p_0$ и $\alpha$ известны).

\AddProb Найдите максимальную температуру идеального газа в процессе, протекающем по закону $P=P_0~-~\alpha~V^2$, 
где $P_0$ и $\alpha$ -- положительные постоянные, $V$ -- объем одного моля.

\AddProb Для идеального газа с заданным показателем адиабаты $\gamma$ найдите уравнение процесса (в координатах $V$, $T$), 
при котором теплоемкость зависит от температуры по закону $c=\xi~T^2$.

\begin{wrapfigure}{r}{4cm}
\includegraphics[scale=0.6]{092005HeatCapacityCylinder.jpg}
\end{wrapfigure}

\AddProb (2005) В расположенном горизонтально цилиндре слева от закрепленного поршня находится один моль идеального газа, 
в правой части цилиндра - вакуум. Цилиндр теплоизолирован от окружающей среды, а пружина, расположенная между поршнем и стенкой, 
находится первоначально в недеформированном состоянии. Поршень освобождают, и после установления равновесия объем, занимаемый газом, 
увеличивается в $\alpha$ раз. Как изменились при этом температура и давление газа? Теплоемкостями цилиндра, поршня и пружины пренебречь. 
Найти теплоемкость газа.

\begin{wrapfigure}{r}{3cm}
\includegraphics{082001GasLawsProcess.jpg}
\end{wrapfigure}

%11
\AddProb (2001) Вычислить молярную теплоемкость $C_P(V)$ идеального газа, совершающего процесс, показанный на рисунке. 
Показатель адиабаты $\gamma$ считать известным.

\chapter{Электричество}
\section{��������������}

%1
\AddProb ��� ��������� ������, ����� ������� $m$ � $M$, �������� ����������� �������� $q$ � ������������ �� ���������� $L$ ���� �� �����. 
������ ���������, � ��� �������� �����������. ����� �������� ������� ����� ������� �� ������� ����������. 
����� �������� ������� ����� ������� �� ���������� $7L$.

\AddProb �������� ����� $q$ ��������� ����� ����� ������������ ����������� ���������������� ������� 
��������� $a$ � $b$ �� ���������� $r$ �� ������ ($a~<~r~<~b$). 
����� ������ �������������� �� ������ ������. ����������� ��� ��������� ���������� ������.

\begin{wrapfigure}{r}{3cm}
\includegraphics[width=3cm]{1003ElectrostaticsBall.jpg}
\includegraphics[width=3cm]{1004ElectrostaticsTetrahedron.jpg}
\end{wrapfigure}

\AddProb ���������� ������������� ��� ������� $R$ �������� �� ��� ����� �� ���������, ��������� �� ������ �� ���������� $h$. 
����� ����, � ������� ������������� ��� �����. �������� ����� ����~$Q$.

\begin{figure}{h}
\includegraphics[scale=0.5]{1005ElectrostaticsTwoPlates.jpg}
\end{figure}

\AddProb ����� ����������� ��������� �� �������� $a$ ���������� �������� � ������������� ���������� ������ $\sigma$. 
� ����� ��������� ������� �������� ����� $q$. ����� ����, � ������� �������� ����� ��������� �� ���� �� ������ ���������.

\AddProb ��� ������� ���������� �������� $1$ � $2$ ����������� �� ���������� $d$ ���� �� �����, 
� ����� ���� �� ���������� $�$ �� �������� $1$ ��������� ���������� �������� � ������� $q$. 
������� �������� ��������� ����������� � ����� ����� $-q$. 
����� ����� ������� �� ����������, ������������ ������� ��������, ���� �������� � ������� $q$ ����������� �� ��������� $x$ � ��������� � �����������~$x_1$.


\section{������������}

\begin{wrapfigure}{r}{4cm}
\includegraphics[scale=0.45]{1001Condensers.jpg}
\end{wrapfigure}

%6
\AddProb � ������������� �����, ������������ �� �������, � ��������� ������ ������� ���� $K$ ���������, ����������� �� �������. 
��������� ����� ������� �� �������. ���������� ��������� ���� ����� ��������� � ����� ������� ����� ����� ���������.

\begin{wrapfigure}{r}{3.5cm}
\includegraphics[scale=0.25]{1002Condensers.jpg}
\end{wrapfigure}

\AddProb ������� � ���, ������ {\Large $\varepsilon$}, ������������ ��������� $C_1$ � $C_2$ � �������� �������������� $R$ ��������� ���, ��� �������� �� �������. 
������� ���������� ������� $Q$, ������������ �� ��������� ����� ������������ �����~$K$.

\AddProb �� ������ ������ ���������� ��� ����� ����������� ������� $C$, ������������ � ��������� ���� $U$ ����� �������������~$R$?

\AddProb ��� ������� �������� � �����, ������������ �� ������� (������������� ���� ���������� ����� $R$). 
������������� ������������ �� ��������, � ����� ����������. ����� ������������ ��������. 
1) ����� ��������� ��� ����� �������� $R_1$. 2) ����� ���������� ������� ��������� �� ���� ����� ����� ��������� ������?

\begin{figure}
	\begin{subfigure}{0.5\textwidth}
	\centering
	\includegraphics[scale=0.5]{1005Condensers.jpg}
	\end{subfigure}
	\begin{subfigure}{0.5\textwidth}
	\centering
	\includegraphics[scale=0.5]{1006Condensers.jpg}
	\end{subfigure}
\end{figure}

\AddProb � ����� �� ������� ����� ����������, � ������������ �� ��������. ���� $K_1$ ��������, �������� ���� $K_2$ �����������. 
1) ����� ���������� ����������� �� �������������? 2) ����� ����� �������� ����� ���� $K_2$ ��� ���������?
\section{Постоянный электрический ток}

\begin{ex}
Найти сопротивления приведенных цепей между точками $A$ и $B$. Сопротивление каждого резистора известно и равно $R$.
\begin{center}
\includestandalone[scale=0.5]{Pictures/1101DirectCurrentABCircuitLine}
\includestandalone[scale=0.5]{Pictures/1101DirectCurrentABCircuitCube}
\includestandalone[scale=0.5]{Pictures/1101DirectCurrentABCircuitSquare}
\includestandalone[scale=0.5]{Pictures/1101DirectCurrentABCircuitStar}
\end{center}
\begin{ans}
$R/3$, $5R/6$, $6R/7$, $7R/15$
\end{ans}
\end{ex}

\begin{ex}
\hspace{0pt} \\
\begin{minipage}{.65\textwidth}
Сопротивления резисторов $R_1 = 1$ Ом, $R_2 = 2$ Ом, $R_3 = 3$ Ом, $R_4 =4$ Ом. 
Напряжение источника тока $U = 1$ В. Найдите ток, который течет через вертикальную перемычку.
\end{minipage}
\begin{minipage}{.35\textwidth}
\centering
\includestandalone[width = 0.8 \textwidth]{Pictures/1102DirectCurrentFourResistors}
\end{minipage}
\begin{ans}
$I = 2/21$ А
\end{ans}
\end{ex}

\begin{ex}
\hspace{0pt} \\
\begin{minipage}{.65\textwidth}
Три одинаковых медных кольца радиуса $r$ соединены так, как показано на рисунке. 
Найдите сопротивление полученной таким образом фигуры, внешнее напряжение подано к точкам $A$ и $B$. 
Удельное сопротивление меди $\rho$, диаметр проволоки~$d$.
\end{minipage}
\begin{minipage}{.35\textwidth}
\centering
\includestandalone[width = 0.75 \textwidth]{Pictures/1103DirectCurrentThreeRings}
\end{minipage}
\begin{ans}
$R = \rho r /16d^2$
\end{ans}
\end{ex}

\begin{ex}
Найдите эквивалентное сопротивление между точками $A$ и $B$ бесконечной цепочки, которая состоит из одинаковых резисторов сопротивлением $R$ каждый.

\centering
\includestandalone{Pictures/1104DirectCurrentFourInfiniteCircuitAB}
\begin{ans}
$(6-\sqrt{3})R/6$
\end{ans}
\end{ex}

\begin{ex}
\hspace{0pt} \\
\begin{minipage}{.65\textwidth}
Из бесконечной проводящей квадратной сетки, каждое звено которой имеет сопротивление $R$, удалили одно звено $AB$. Найдите сопротивление сетки между точками $A$ и $B$.
\end{minipage}
\begin{minipage}{.35\textwidth}
\centering
\includestandalone[scale = 0.7]{Pictures/1105DirectCurrentNet}
\end{minipage}
\begin{ans}
$R$
\end{ans}
\end{ex}

\begin{ex}
Имеется $n$ клемм, каждая из которых соединена со всеми остальными клеммами одинаковыми проводниками сопротивлением $R$. Найдите сопротивление между любыми двумя клеммами.
\begin{ans}
$R_1 = 2R/n$
\end{ans}
\end{ex}

\begin{ex}
Электрический чайник имеет две обмотки. При включении одной из них чайник вскипает через 10 мин, при включении другой -- через 15 мин. 
Через какое время чайник вскипит, если эти две обмотки включить вместе параллельно, последовательно?
\begin{ans}
$t_1 = 6$ мин, $t_2 = 25$ мин
\end{ans}
\end{ex}

\begin{ex}
\hspace{0pt} \\
\begin{minipage}{.65\textwidth}
На рисунке представлен график зависимости силы тока от напряжения на нелинейном резисторе. 
Определите силу тока в цепи при подключении этого резистора к источнику тока с напряжением 10 В и добавочным сопротивлением 100 Ом.
\end{minipage}
\begin{minipage}{.35\textwidth}
\centering
\includestandalone[width = 0.95 \textwidth]{Pictures/1108DirectCurrentVDR}
\end{minipage}
\begin{ans}
$I = 0,06$ А
\end{ans}
\end{ex}

\begin{ex}
\hspace{0pt} \\
\begin{minipage}{.65\textwidth}
На рисунке приведен график зависимости напряжения на разрядном промежутке дугового разряда от тока. Дугу подключают  к источнику постоянного напряжения последовательно с резистором. 
При каком максимальном значении сопротивления резистора дуга может гореть при напряжении источника $U =85$~В?
\end{minipage}
\begin{minipage}{.35\textwidth}
\centering
\includestandalone[width = 0.85 \textwidth]{Pictures/1109DirectCurrentArcDischarge}
\end{minipage}
\begin{ans}
$R = 5$ Ом
\end{ans}
\end{ex}

\begin{ex}
\hspace{0pt} \\
\begin{minipage}{.65\textwidth}
Схема, изображенная на рисунке, состоит из двух одинаковых резисторов $R_2$ и $R_3$ сопротивлением $R$ каждый и двух одинаковых нелинейных резисторов $R_1$ и $R_4$, вольтамперная характеристика которых имеет вид $U = \alpha I^2$. При каком напряжении источника питания $U_0$ сила тока через гальванометр равна нулю?
\end{minipage}
\begin{minipage}{.35\textwidth}
\centering
\includestandalone[width = 0.9 \textwidth]{Pictures/1110DirectCurrentResistorsAndGalvanometor}
\end{minipage}
\begin{ans}
$U_0 = 2R^2/\alpha$
\end{ans}
\end{ex}

\begin{ex}
(2018) Проволочный предохранитель перегорает при напряжении 300 В. При каком напряжении будет перегорать предохранитель, если его длину увеличить в 3 раза, a диаметр -- в 2 раза?
\begin{ans}
636 В
\end{ans}
\end{ex}
\section{Закон Био-Савара-Лапласа}

%1
\AddProb Ток $I$ течет по длинному прямому проводнику, сечение которого имеет форму тонкого полукольца радиуса $R$. Найти магнитную индукцию на оси~$O$.

\begin{wrapfigure}{r}{2.5cm}
\includegraphics[scale=0.4]{1202BiotSavartLawConductor.jpg}
\end{wrapfigure}

\AddProb Найти модуль и направление силы, действующей на единицу длины тонкого проводника с током $I$ в точке $O$, 
если проводник изогнут так, как показано на рисунке.


\section{Сила Ампера}

\begin{wrapfigure}{r}{4cm}
\includegraphics[width = 4cm]{RodInMagneticField.png}
\end{wrapfigure}
\AddProb (2016) На рисунке представлена модель электродвигателя. Замкнутый контур образован двумя вертикальными рейками, между концами которых включен источник постоянного тока с ЭДС {\Large $\varepsilon$}, a другие концы замкнуты перемычкой сопротивлением $R$ и длиной $L$. Перемычка за счет скользящих контактов может без трения скользить вдоль реек. Контур находится в однородном магнитном поле с индукцией $B$, направленной горизонтально. Известно, что если к перемычке подвесить груз массы $M$, она будет в состоянии равновесия. 1) Определите массу $m$ перемычки. 2) Определите установившуюся скорость ненагруженной перемычки. Сопротивлением реек и внутренним сопротивлением источника пренебречь.

\AddProb (2003) Металлический стержень массой $m$ и длиной $L$ подвешен на двух легких проводах длиной $l$ в магнитном поле с индукцией $B$, 
вектор которой направлен вертикально. К точкам крепления проводов подключен конденсатор емкостью $C$, заряженный до напряжения $U$. 
Сопротивление стержня и проводов пренебрежимо мало. Найти максимальный угол отклонения проводов от вертикали, 
если разрядка конденсатора происходит за очень малое время.

\begin{wrapfigure}{r}{4.5cm}
\includegraphics[scale=1]{122010AmperesForceLawTwoConductors.jpg}
\end{wrapfigure}

\AddProb (2010) Посередине между двумя жестко закрепленными проводниками с током на расстоянии $a$ расположен груз массы $m$, 
представляющий собой цилиндрическую железную трубку длиной $l_1$ и укрепленный с помощью упругих растяжек длиной $l$. 
Магнитная проницаемость железа $\mu$. Внутри растяжек установлен еще один проводник. По всем трем проводникам течет ток $J$. 
Определите собственную частоту свободных колебаний груза, считая, что в процессе колебаний натяжение растяжек $T_0$ не изменяется. 
Определить зависимость критического значения силы тока от натяжения растяжек, считая критическим значением такое значение, 
при котором колебания в системе невозможны.

\begin{wrapfigure}{r}{3cm}
\includegraphics[width = 3cm]{AmperGenerator.png}
\end{wrapfigure}
\AddProb (2017) Одна из моделей генератора постоянного тока представляет собой проводящий диск, который вращается в однородном магнитном поле с индукцией,
направленной перпендикулярно плоскости вращения диска. Если концы некоторого проводника сопротивлением $R$ присоединить к центру диска и через скользящий контакт к его ободу, то в цепи возникнет электрический ток. 1) Объясните
возникновение электрического тока и найдите силу тока, если радиус диска $r = 10$ см, частота вращения диска $\nu = 40$ об/с, индукция магнитного поля $B = 0,1$ Тл, сопротивление нагрузки $R = 0,5$ Ом. 2) Какая мощность затрачивается для поддержания вращения диска? 3) Какой момент силы относительно оси вращения нужно прикладывать к диску? Сопротивлениями диска и контактов пренебречь.


\section{Теорема о циркуляции индукции магнитного поля}

\begin{wrapfigure}{r}{3cm}
\includegraphics[scale=0.25]{1205AmperesCircuitalLawDisc.jpg}
\end{wrapfigure}

\AddProb Однородный диэлектрический диск массой $m$ радиуса $R$, равномерно заряженный с полным зарядом $q$, 
помещен в однородное магнитное поле с индукцией $B$. Какую угловую скорость получит диск, если выключить магнитное поле?

\begin{wrapfigure}{r}{4cm}
\includegraphics[scale=0.5]{1206AmperesCircuitalLawSphere.jpg}
\end{wrapfigure}

%6
\AddProb По поверхности жесткой непроводящей однородной сферы массой $m$ равномерно распределен заряд $q$. 
Сфера может свободно вращаться вокруг своей вертикальной оси. В начальный момент сфера покоилась, а магнитное поле было равно нулю. 
Найти, как меняется со временем угловая скорость сферы при включении однородного магнитного поля, 
сонаправленного с осью вращения сферы и меняющегося во времени по заданному закону $B(t)$.


\section{Электромагнитная индукция}

\begin{wrapfigure}{r}{4cm}
\includegraphics[scale=0.5]{122002EMIFrame.jpg}
\end{wrapfigure}

\AddProb (2002) Прямоугольная рамка со сторонами $a$ и $b$ находится в одной плоскости с прямым проводником, 
по которому течет ток $I$, на расстоянии $L$ от него. Какой импульс получит рамка при выключении тока в проводе, 
если активное сопротивление рамки равно $R$, а реактивным сопротивлением ее можно пренебречь? 
Считать, что за время передачи импульса рамка заметно не перемещается.

\AddProb (2001) На расстоянии $a$ и $b$ от длинного прямого провода с током $I$ расположены два параллельных ему провода, 
замкнутые с одной стороны сопротивлением $R$. По проводам без трения перемещаются с постоянной скоростью $v$ стержень-перемычку. 
Пренебрегая сопротивлением проводов, стержня и контактов, найдите силу, необходимую для поддержания постоянства скорости. 
На каком расстоянии от ближнего провода нужно приложить силу, чтобы избежать вращения стержня?

\begin{figure}[!h]
\includegraphics[scale=0.9]{122001EMIWiresAndBar.jpg}
\end{figure}

\AddProb (2004) Металлический стержень массы $m$ и длины $L$ подвешен горизонтально на двух легких проводах длиной $h$ в магнитном поле, 
индукция которого $B$ направлена вертикально вниз. К точкам крепления проводов подключен конденсатор емкостью $C$. 
Стержень вывели из положения равновесия и отпустили. Определить период малых колебаний стержня $T$. Сопротивлением стержня и проводов пренебречь.

\begin{wrapfigure}{r}{2cm}
\includegraphics[scale=0.5]{121998EMIConductorAC.jpg}
\end{wrapfigure}

\AddProb (1998) По двум вертикальным рейкам, соединенным внизу сопротивлением $R$ и вверху источником с ЭДС {\Large $\varepsilon$} 
и внутренним сопротивлением $r$, без трения скользит проводник $AC$, длина которого $L$, масса $m$. 
Система находится в однородном магнитном поле с индукцией $B$, направленной за рисунок. 
Найдите установившуюся скорость проводника в поле силы тяжести, пренебрегая трением и сопротивлением реек и проводника.

\begin{wrapfigure}{r}{4cm}
\includegraphics[scale=0.25]{1211EMIHillAndBridge.jpg}
\end{wrapfigure}

%11
\AddProb По двум параллельным металлическим направляющим, наклоненным под углом $\alpha$ к горизонту и 
расположенным на расстоянии $l$ друг от друга, может скользить без трения металлическая перемычка массой $m$. 
Направляющие замкнуты снизу на незаряженный конденсатор емкостью $C$, и вся конструкция находится в магнитном поле, 
индукция которого $B$ направлена по вертикали. В начальный момент перемычку удерживают на расстоянии $l$ от основания "горки". 
Определите время $t$, за которое перемычка достигнет основания "горки" после того, как ее отпустят. 
Омическим сопротивлением и индуктивностью контура пренебречь.

\AddProb (2001) Две параллельные медные шины наклонены к горизонту под углом  $\alpha$. 
По ним скользит под действием силы тяжести медная перемычка массы $m$. Шины замкнуты катушкой с индуктивностью $L$. 
Система находится в однородном магнитном поле индукции $B$, перпендикулярном плоскости, в которой движется перемычка. 
Коэффициент трения перемычки о шины равен $\mu$. Каков будет характер движения перемычки? Сопротивлением шин, перемычки и катушки пренебречь.

\begin{wrapfigure}{r}{4.5cm}
\includegraphics[scale=1]{122009EMICoin.jpg}
\end{wrapfigure}

\AddProb (2009) Медная монета массой $m$ радиусом $R$ и толщиной $d$ движется в поле силы тяжести в однородном магнитном поле $B$. 
Вектор индукции магнитного поля направлен вдоль оси монеты и перпендикулярно ускорению свободного падения. Найти ускорение монеты.

\begin{wrapfigure}{r}{5cm}
\includegraphics[scale=0.5]{1214EMIRailsAndBridge.jpg}
\end{wrapfigure}

\AddProb Параллельные рельсы длиной $2L$ закреплены на горизонтальной плоскости на расстоянии $l$ друг от друга. 
К их концам подсоединены две одинаковые батареи с ЭДС {\Large $\varepsilon$}. На рельсах лежит перемычка массы $m$, 
которая может поступательно скользить вдоль них. Вся система помещена в однородное вертикальное магнитное поле с индукцией $B$. 
Считая, что сопротивление перемычки равно $R$, а сопротивление единицы длины каждого из рельсов равно $\rho$, найдите период малых колебаний, 
возникающих при смещении перемычки от положения равновесия, пренебрегая затуханием, внутренним сопротивлением источников, 
сопротивлением контактов, а также индуктивностью цепи.

\AddProb (2015) Проводящая квадратная рамка пересекает область однородного магнитного поля с шириной $d$, линии напряженности которого перпендикулярны плоскости рамки. При этом скорость рамки, равная $v_0$ до входа в магнитное поле уменьшается в 2 раза Масса рамки равна $m$, сопротивление рамки — $R$, величина вектора магнитной индукции - $B$. 1) Объясните, почему в рамке при пересечении магнитного поля выделяется тепло и найдите его. 2) Определите длину стороны рамки $a$ предполагая, что $a<d$. 3) Определите длину стороны рамки $a$ предполагая, что $a>d$.

\section{Движение заряженных частиц}

\AddProb Заряженная частица массы $m$ влетает в магнитное поле $B$ под углом $\alpha$ со скоростью $v$. По какой траектории движется частица? Каков пространственный период витка (шаг спирали)?

%Зональные студенческие олимпиады Ижевска
\AddProb По обмотке длинного цилиндрического соленоида радиуса $R$ протекает постоянный ток, создающий внутри соленоида однородное магнитное поле с индукцией $B$. Между витками соленоида в него влетает по радиусу (перпендикулярно оси соленоида) электрон со скоростью $v$. Отклоняясь в магнитном поле, электрон спустя некоторое время покинул соленоид. Определите время движения внутри соленоида.

\AddProb (2008) Электронно-лучевая трубка помещена в однородное магнитное поле, напряженность $H$ которого перпендикулярна плоскости экрана. Электроны влетают в электронно-лучевую трубку из электронной пушки с составляющей скорости $u$ вдоль оси трубки и составляющей скорости $v_0$ перпендикулярно оси. При какой длине $L$ трубки все электроны фокусируются в одной точке экрана?

\AddProb (2012) Две заряженные частицы движутся в однородном магнитном поле $B$, причем $q_1/m_1 = q_2/m_2$. Написать уравнения движения центра масс и уравнение относительного движения.

%16
\AddProb (2013) Заряд $q$ движется в поле магнитного монополя $\vec{B} = \alpha \vec{r}/r^3$. Найдите интеграл движения, следующий из закона изменения момента импульса заряда.

\AddProb Частица c зарядом $q$ и массой $m$ движется с начальной скоростью $v_0$ в вязкой среде в поперечном магнитном поле с индукцией $B$. Сила сопротивления $\vec{F} = -\gamma \vec{v}$, где $\gamma$ - константа. На каком расстоянии от начальной точки частица остановится?

\AddProb (2007) Частица c зарядом $q$ и массой $m$ движется в постоянных однородных скрещенных полях $\vec{E} \bot \vec{H}$ в среде с малым линейным сопротивлением $\vec{F} = -\gamma \vec{v}$. Найти скорость частицы вдоль поля $\vec{E}$, усредненную по периоду.

%Зональные студенческие олимпиады Ижевска, Черепанов
\AddProb На магнитный барьер, задаваемый в пространстве статическим магнитным полем $\vec{B} = \left(0, 0, \frac{B_0}{\cosh^2(ky)}\right)$, где $k$ - константа, из бесконечности налетает протон с начальной скоростью $\vec{v}_{-\infty} = (0, v_0, 0)$, $\vec{r}_{-\infty} = (0, -\infty, 0)$. Оцените минимальную скорость, которую должен иметь протон, чтобы преодолеть барьер и уйти на бесконечность. 

\begin{wrapfigure}{r}{4cm}
\includegraphics[width = 4cm]{ElectronInVacuumDiode.png}
\end{wrapfigure}
\AddProb (2018) Вакуумный диод представляет собой две металлические пластины -- катод и анод. Между пластинами имеется однородное магнитное поле с индукцией $B$, параллельной плоскости пластин (направленной из плоскости чертежа). Расстояние между пластинами $d$. Из катода вылетают электроны. 1) При каких начальных скоростях все электроны не смогут достичь анода при $U = 0$? 2) При каких напряжениях $U$ все электроны не смогут достичь анода при нулевой начальной скорости?

\AddProb (2005)  Магнетрон -- это прибор, состоящий из нити накала радиуса $a$ и коаксиального цилиндрического анода радиуса $b$, которые находятся в однородном магнитном поле параллельном нити. Между нитью и анодом приложена ускоряющая разность потенциалов $U$. Найти минимальное значение индукции магнитного поля $B$, при котором электроны, вылетающие с нулевой начальной скоростью из нити, не будут достигать анода.

\AddProb (2014) Незаряженная неподвижная частица распалась в однородном магнитном поле с индукцией $B$ на две частицы с массами $m_1$ и $m_2$ и зарядами $+q$ и $-q$. Найдите время, через которое произойдет соударение частиц. Кулоновским взаимодействием между частицами пренебречь.

\section{Колебательный контур}

\begin{wrapfigure}{r}{3.5cm}
\includegraphics[scale=0.5]{1219OscillatoryCircuitCCL.jpg}
\end{wrapfigure}

\AddProb Батарея из двух последовательных соединенных конденсаторов емкостью $C$ каждый заряжена до напряжения $U$ 
и в начальный момент времени подключена к катушке индуктивностью $L$, так что образовался колебательный контур. 
Спустя интервал времени $\tau$ один из конденсаторов пробивается, и сопротивление между обкладками становится равным нулю. 
Найдите амплитуду колебаний заряда на непробитом конденсаторе.

\begin{wrapfigure}{r}{4cm}
\includegraphics[scale=0.4]{1220OscillatoryCircuitC1C2L.jpg}
\end{wrapfigure}

\AddProb Два конденсатора одинаковой электроемкости $C_1~=~C_2~=~C$ и катушка индуктивности $L$ соединены так, как показано на рисунке. 
В начальный момент времени ключ разомкнут, конденсатор $C_1$ заряжен до разности потенциалов $U$, а конденсатор $C_2$ не заряжен, 
сила тока в катушке равна нулю. Определите максимальное значение силы тока в катушке после замыкания цепи и период электромагнитных колебаний в цепи. 

\begin{wrapfigure}{r}{3.5cm}
\includegraphics[scale=0.5]{1221OscillatoryCircuitCLR.jpg}
\end{wrapfigure}

%21
\AddProb В схеме, изображенной на рисунке, в некоторый момент времени замыкают ключ $K$, и конденсатор емкостью $C$, 
имеющий первоначальный заряд $q_0$, начинает заряжаться через катушку индуктивности $L$. 
Когда ток разряда достигает максимального значения, ключ $K$ вновь размыкают. Найти заряд $Q$, который протечет через резистор $R$. 
Сопротивление диода в прямом направлении много меньше $R$, в обратном - бесконечно велико.

\begin{wrapfigure}{r}{3.5cm}
\includegraphics[scale=1]{122011OscillatoryCircuitTriangleCCCLLL.jpg}
\end{wrapfigure}

\AddProb (2011) Электрический контур представляет собой треугольник, каждая сторона которого содержит емкость $C$, 
а вершины соединены с общей центральной точкой индуктивностями $L$. 
Пренебрегая сопротивлением и взаимной индуктивностью, найдите частоту возможных колебаний.


\section{Переменный электрический ток}

\begin{wrapfigure}{r}{2cm}
\includegraphics[scale=0.4]{1223AlternatingCurrentCLR.jpg}
\end{wrapfigure}

\AddProb В изображенной на рисунке электрической цепи определите частоту приложенного переменного напряжения, 
при которой переменный ток через сопротивление не зависит от значения $R$. Индуктивность $L$ и емкость $C$ считать известными.

\AddProb При каком условии амплитуда тока в цепи зависит только от амплитуды приложенного напряжения, 
но не от его частоты? Индуктивность $L$, и емкость $C$, и сопротивление $R$ считать известными.

\begin{figure}[!h]
\includegraphics[scale=0.5]{1224AlternatingCurrentCCLR.jpg}
\end{figure}

\AddProb В приведенной на рисунке схеме в момент $t$~ =~0 замыкают ключ $K$. Найти зависимость от времени тока $I$, 
текущего через источник синусоидальной ЭДС {\Large $\varepsilon$}~=~{\Large $\varepsilon_o$}\,$\sin\,\omega\,t$.
%$\varepsilon\,=\,\varepsilon_0\,\sin\,\omega\,t$.  
Параметры контура связаны соотношением $R\,=\,\sqrt{L/C}$.

\begin{figure}[!h]
\includegraphics[scale=0.5]{1225AlternatingCurrentCLRR.jpg}
\end{figure}

\chapter{Колебания}
\section{Механические колебания}
%TODO разделы
%TODO задача краевой олимпиады 2007


%-----------------Пружинный маятник---------------------
%Ландау
\begin{ex}
\hspace{0pt} \\
\begin{minipage}{.65\textwidth}
Найдите частоту колебаний точки с массой $m$, способной двигаться по прямой и прикрепленной к пружине, другой конец которой закреплен в точке $A$ на расстоянии $l$ от прямой. Пружина, имея длину $l$, натянута с силой $F$.
\end{minipage}
\begin{minipage}{.35\textwidth}
\centering
\includestandalone[width = 0.95 \textwidth]{Pictures/1303OscillationsSpringAndLine}
\end{minipage}
\begin{ans}
$\omega = \sqrt{F/ml}$
\end{ans}
\end{ex}

%Ландау
\begin{ex}
\hspace{0pt} \\
\begin{minipage}{.65\textwidth}
Найдите частоту колебаний точки с массой $m$, способной двигаться по окружности радиуса $r$ и прикрепленной к пружине, другой конец которой закреплен в точке $A$, кратчайшее расстоянии то точки $A$ до окружности равно $l$. Пружина, имея длину $l$, натянута с силой $F$.
\end{minipage}
\begin{minipage}{.35\textwidth}
\centering
\includestandalone[width = 0.95 \textwidth]{Pictures/1304OscillationsSpringAndCircle}
\end{minipage}
\begin{ans}
$\omega = \sqrt{\frac{F(r+l)}{rml}}$
\end{ans}
\end{ex}

\begin{ex}
\hspace{0pt} \\
\begin{minipage}{.65\textwidth} 
(2007) С какой частотой $\omega_0$ будет совершать малые вертикальные колебания в поле тяжести груз массы $m$, подвешенный на двух одинаковых пружинах жесткости $k$, образующих в равновесии углы $\beta$ с вертикалью?
\end{minipage}
\begin{minipage}{.35\textwidth}
\centering
\includestandalone[width = 0.95 \textwidth]{Pictures/132007OscillationsTwoSpringsAndWeight}
\end{minipage}
\begin{ans}
$\omega = \cos \beta \sqrt{2k/m}$
\end{ans}
\end{ex}

%Иродов-3.51
\begin{ex}
\hspace{0pt} \\
\begin{minipage}{.65\textwidth} 
Найти круговую частоту малых колебаний тонкого стержня массы $m$ и длины $l$ вокруг горизонтальной оси, проходящей через точку $O$. 
Жесткость пружины $k$. В положении равновесия стержень вертикален.
\end{minipage}
\begin{minipage}{.35\textwidth}
\centering
\includestandalone{Pictures/SpringRodOsc}
\end{minipage}
\begin{ans}
$\omega = \sqrt{3g/2l+3k/m}$
\end{ans}
\end{ex}

\begin{ex}
(2016) Студент университета Пружинкин, увлекающийся физическим экспериментом (и не очень любящий теорию) смонтировал в физической лаборатории маятник. Маятник представлял собой очень легкий шарнирно закрепленный стержень длины $l = 50$ см (шарнир в нижней точке стержня), на котором были закреплены два одинаковых шарика c массами по $m = 0,9$ кг: один шарик -- в середине стержня, a другой нa верхнем его конце. Между стержнем и вертикальной неподвижной стенкой на высоте $h=3l/4$ от шарнира студент закрепил горизонтальные пружинки различной жесткости. В положении равновесия деформация пружинок была равна нулю. Для того, чтобы построить экспериментальный график зависимости периода малых колебаний от жесткости, студент изготовил для опытов пять пружинок с известными жесткостями $k = 10, 20, 30, 40, 50$ Н/м. Выведите выражение для периода колебаний маятника и предскажите результаты опытов Пружинкина.
\begin{center}
\includestandalone{Pictures/InvertedPenduumWithSpring}
\end{center}
\begin{ans}
$T=2\pi \sqrt{\frac{20ml}{9kl-24mg}}$, при $k>24mg/9l$
\end{ans}
\end{ex}

\begin{ex}
(2015) На гладкой горизонтальной поверхности находится грузик, прикрепленный двумя одинаковыми пружинами к стенкам (вид сверху на рисунке слева). 
В положении равновесии грузика пружины имеют одинаковое растяжение~$\Delta l$. 1) В первом случае груз совершает малые колебания вдоль оси~$x$. 
Как зависит период этих колебаний от величины~$\Delta l$?   2) Во втором случае траектория грузика, совершающего малые колебания, изображена на рисунке справа (в увеличенном виде).
Определите $\Delta l$, если длина пружин в нерастянутом состоянии $l$~=~15 см. Во всех случаях выполняется закон Гука.
\begin{center}
\includestandalone{Pictures/132015OscillationsLissajousCurve}
\end{center}
\begin{ans}
$\omega_x = \sqrt{2k/m}$, $\omega_y=\sqrt{\frac{2k\Delta l}{m(l+\Delta l)}}$, $\Delta l = l/3$
\end{ans}
\end{ex}

%Ижевск
\begin{ex}
\hspace{0pt} \\
\begin{minipage}{.65\textwidth}
Схема динамического поглотителя колебаний представлена на рисунке. На первую массу действует гармоническая сила $F(t)~=~F_0\,\sin\,\omega\,t$. 
При каких условиях амплитуда вынужденных колебаний первой массы будет равна нулю?
\end{minipage}
\begin{minipage}{.35\textwidth}
\centering
\includestandalone{Pictures/1302OscillationsDynamicAbsorbentOfOscillations}
\end{minipage}
\begin{ans}
$\omega = \sqrt{k_2/m_2}$
\end{ans}
\end{ex}

%Черепанов
\begin{ex}
Груз массой $m$ прикреплен к пружине жесткостью $k$, а пружина к точке подвеса. 
Под действием внешней силы точка подвеса колеблется в вертикальном направлении по закону $x_p~=~A~\cos\,\omega t$. 
Какова амплитуда установившихся колебаний груза в вязкой среде, если сила сопротивления пропорциональна скорости $(F~=~-bv)$?
\begin{ans}
$x^2 = A^2\frac{\omega_0^4+4\beta^2\omega^2}{(\omega_0^2-\omega^2)^2+4\beta^2\omega^2}$, $\omega_0 = \sqrt{k/m}$, $\beta = b/2m$
\end{ans}
\end{ex}

%-------------Колебания газа в банке---------------
%Бутиков
\begin{ex} Расположенный горизонтально цилиндрический сосуд объема $V$, заполненный $\nu$ молями идеального газа, 
разделен поршнем массы $m$, который может двигаться без трения. В равновесии поршень находится посередине цилиндра. 
При малых смещениях из положения равновесия поршень совершает колебания. Найдите зависимость частоты этих колебаний от температуры, 
считая процесс изотермическим. Площадь поперечного сечения трубки равна~$S$.
\begin{ans}
$\omega^2=\frac{2\nu R S^2}{mV^2}T$
\end{ans}
\end{ex}

%Ижевск
\begin{ex}
\hspace{0pt} \\
\begin{minipage}{.65\textwidth} 
В длинной вертикальной цилиндрической трубке, закрытой с нижнего конца, может ходить без трения поршень, 
масса $M$ которого велика по сравнению с массой газа, заключенного внутри трубки. В положении равновесия расстояние между поршнем и дном трубки равно $L$. 
Определить период малых колебаний, которые возникнут при отклонении поршня от положения равновесия, в предположении, что они являются изотермическими, 
а газ идеальным. Площадь поперечного сечения трубки равна $S$, атмосферное давление равно $p_0$. Рассмотреть предельный случай, когда $p_0$~=~0.
\end{minipage}
\begin{minipage}{.35\textwidth}
\centering
\includestandalone{Pictures/1308OscillationsPistonInVerticalTube}
\end{minipage}
\begin{ans}
$T=2\pi\sqrt{\frac{ML}{p_0S + Mg}}$
\end{ans}
\end{ex}

%-----------Колебания жидкости-------------
%Иродов-3.20
\begin{ex}
\hspace{0pt} \\
\begin{minipage}{.65\textwidth} 
Идеальная жидкость объема $V$ налита в изогнутую трубку с площадью сечения канала $S$. Найти период малых колебаний жидкости.
\end{minipage}
\begin{minipage}{.35\textwidth}
\centering
\includestandalone{Pictures/1310OscillationsPerfectLiquid}
\end{minipage}
\begin{ans}
$T=\pi \sqrt{2V/Sg}$
\end{ans}
\end{ex}

%Иродов-3.21
\begin{ex}
Трубка высотой $H$ наполнена жидкостью и соединена с наклонной трубкой (угол наклона к вертикали $\alpha$). Каков будет период колебаний жидкости в такой системе?
\begin{ans}
$T=2\pi \sqrt{H/g(1+\cos \alpha)}$
\end{ans}
\end{ex}

\begin{ex}
\hspace{0pt} \\
\begin{minipage}{.65\textwidth} 
(2006) В тонкой трубке может скользить без трения веревка длиной $L$. В начальный момент времени веревка находится в левом колене. 
Определить период колебаний $T$ веревки. Жесткостью веревки на изгиб пренебречь. 
Каким будет период колебаний $T_1$, если расстояние между вертикальными коленами трубки увеличить с $L$ до $2L$?
\end{minipage}
\begin{minipage}{.35\textwidth}
\centering
\includestandalone{Pictures/132006OscillationsRopeInTube}
\end{minipage}
\begin{ans}
$T=2\pi\sqrt{L/g}$, $T=2(\pi+1)\sqrt{L/g}$
\end{ans}
\end{ex}

%-----------Затухающие колебания-------------

% Иродов-3.88
\begin{ex}
\hspace{0pt} \\
\begin{minipage}{.65\textwidth} 
Тонкий однородный диск массы $m$ и радиуса $R$, подвешенный в горизонтальном положении к упругой нити, совершает крутильные колебания в жидкости. 
Момент упругих сил со стороны нити $M~=~\alpha\,\varphi$.  Сила сопротивления на единицу поверхности $F~=~\eta\,v$. Найти частоту малых колебаний.
\end{minipage}
\begin{minipage}{.35\textwidth}
\centering
\includestandalone{Pictures/1309OscillationsDisc}
\end{minipage}
\begin{ans}
$\omega = \sqrt{2\alpha/mR^2-(\pi \eta R^2/m)^2}$
\end{ans}
\end{ex}

%Черепанов
\begin{ex}
(2004) На горизонтальной плоскости с коэффициентом трения $\mu$ лежит брусок массы $m$, который пружиной жесткости $k$ соединен с вертикальной стенкой. Брусок сместили на расстояние $9\mu mg /2k$ и отпустили. Найти число колебаний бруска до остановки.
\begin{ans}
одно колебание
\end{ans}
\end{ex}

%------------Колебания в поле силы тяжести------------
\begin{ex}
\hspace{0pt} \\
\begin{minipage}{.65\textwidth} 
Найти период малых колебаний плоского маятника, точка подвеса которого с массой $M$ находится на гладкой горизонтальной прямой 
(масса маятника $m$ и его длина $L$).
\end{minipage}
\begin{minipage}{.35\textwidth}
\centering
\includestandalone{Pictures/1314OscillationsPlanePendulum}
\end{minipage}
\begin{ans}
$T=2\pi \sqrt{\frac{L}{g(1+m/M)}}$
\end{ans}
\end{ex}

%Ижевск
\begin{ex}
Дана проволочная вешалка, которая качается с малой амплитудой в плоскости рисунка относительно заданных положений равновесия. 
В положениях а и б длинная сторона вешалки расположена горизонтально. Две другие стороны равны между собой. 
Во всех трех случаях (а, б, в) возникают колебания с одинаковыми периодами. Где расположен центр масс вешалки?
\begin{center}
\includestandalone{Pictures/1315OscillationsHanger}
\end{center}
\begin{ans}
На расстоянии 5 см от вершины треугольника
\end{ans}
\end{ex}

\begin{ex}
\hspace{0pt} \\
\begin{minipage}{.65\textwidth}
(2013) Точка подвеса математического маятника движется с постоянным ускорением $a$ (лежащим в плоскости колебаний маятника) в горизонтальном направлении. Найти закон движения маятника.
\end{minipage}
\begin{minipage}{.35\textwidth}
\centering
\includestandalone{Pictures/132013OscillationsMathematicalPendulum}
\end{minipage}
\begin{ans}
$\alpha(t) = A \cos(\sqrt{\sqrt{a^2+g^2}/l}t) + \alpha_0$, $\alpha_0 = \arctan(a/g)$
\end{ans}
\end{ex}

\begin{ex}
(2017) Один конец однородного стержня длиной $L=1$ м закреплен на горизонтальной оси $O$ (перпендикулярной плоскости чертежа), вокруг которой он может вращаться без трения. Другой конец стержня свободно лежит на опоре $A$, при этом стержень горизонтали. Свободный конец стержня приподнимают на высоту $h = 1$ см относительно исходного положения, поворачивая стержень
вокруг оси $O$, и отпускают бы начальной скорости. 1) Определите скорость
свободного конца стержня при ударе об опору $A$. 2) Определите частоту ударов
стержня об опору $A$, считая удары абсолютно упругими и мгновенными.
Ускорение свободного падения $g$ = 10 м/с\textsuperscript{2}. Трением пренебречь.
\begin{center}
\includestandalone{Pictures/HittingRod}
\end{center}
\begin{ans}
1) $v=\sqrt{3gh}$; 2) $\nu = 0.25\sqrt{3g/h }$
\end{ans}
\end{ex}

\begin{ex}
(2010) На абсолютно гибкой нити подвешен маятник. Расстояния до точки подвеса равны $S_1$ и $S_2$. 
Статический прогиб в точке подвеса маятника равен $y_0$, а длина маятника $l$, его масса $m$. 
Принять, что при $t$~=~0 точка подвеса имеет вертикальное смещение $A$, а ее скорость $v_0$ равна нулю; маятник отклонен на угол $\varphi_0$. 
Считая, что натяжение нити $T = T_0 = \text{const}$, вывести дифференциальное уравнение малых колебаний маятника.
\begin{center}
\includestandalone{Pictures/132010OscillationsPendulumOnThread}
\end{center}
\begin{ans}
$ml \varphi^{\prime \prime} + (mg-m \omega^2 A \cos \omega t) \varphi = 0$, $\omega^2 = T_0(S_1+S_2)/m S_1 S_2$
\end{ans}
\end{ex}

\begin{ex}
(2014) В центре плоского конденсатора удерживают диполь в положении 1, указанном на рисунке. 
Диполь представляет собой два маленьких заряженных шарика с зарядами $q$ и $-q$ и массами $m$ каждый, 
соединенных невесомым непроводящим стержнем длины $L$. Разность потенциалов между пластинами конденсатора равна $U$, 
расстояние между пластинами равно $d$ (много меньше размеров пластин). Диполь поворачивают вокруг его центра и переводят в положение 2, 
которое является устойчивым положением равновесия диполя. 
1) Определите работу электрического поля при повороте диполя из положения 1 в положение 2. 
2) Определите период малых крутильных колебаний диполя вокруг положения 2. 
3) Какую минимальную скорость нужно сообщить диполю в положении 2 параллельно пластинам для того, чтобы диполь улетел из конденсатора? 
Силу тяжести не учитывать.
\begin{center}
\includestandalone{Pictures/132014OscillationsCondenserAndDoublet}
\end{center}
\begin{ans}
1) $A=qUL/d$; 2) $T=2\pi\sqrt{\frac{mLd}{2qU}}$; 3) $v_0 = \sqrt{\frac{qUL}{md }}$
\end{ans}
\end{ex}

\begin{ex}
Молекулу одноатомного газа массы $m$, совершающую колебания около некоторого положения равновесия с амплитудой $a$ и частотой $\omega$, 
в первом приближении можно считать линейным гармоническим осциллятором. 
Найти $f(x)$ -- функцию распределения вероятностей значений координаты $x$ такого осциллятора, среднее значение координаты $\langle x \rangle$, 
среднее значение $\langle U \rangle$ потенциальной энергии осциллятора.
\end{ex}

\part{Чемпионаты по физике для студентов Физфака}

\chapter{Студенческий чемпионат по физике 2006 года}
\section{I ����}

\AddProb �� ������� �������������� ����������� ����� ��������� ����� ����� $m$ � ������� 
����� ����� $M$ ������ $H$. ����� ����������� �������� $v$ ���� ������� �����, ����� ��� 
������ ���������� ������?

\AddProb ������� ����� $M$ � ����� $L$ ������������ �����������, ��� ��� �� ������ ����� 
�������� �����. ���� ���������, � ��� ������ �� ����. ������ ����� ��������� �����, ����� 
����� ������� ����� $x$ ��� ����� �� ����?

\AddProb ������� �������������� ����� �������� ���� � ���������� ��������� $v_0$. � ������ $h$ 
��������� ��� ��������� �������� �����, ������� �������� ������������ �� �����. ������ 
����� ������ � ����� ��������, ����� ����������� �������� ������ �� ������� � ��������� 
�� ������.


\section{II ����}

\AddProb �� �������������� ��������� ����� ����� ����� �������~$V$. ������ �� 
����� ���������� ������ ������~$M$ ���, ��� ����� ������ ����������. ������� 
����������~$h$ ����� ������� � ����������, ���� ����������� �������������� 
��������� ����� $\sigma$ � ��� ����������� ��������� �� ����������� ������.

\AddProb ������� ����� $M$ � ����� $L$ ������������ �����������, ��� ��� �� ������ ����� 
�������� �����. ���� ���������, � ��� ������ �� ����. ������ ����� ��������� �����, ����� 
����� ������� ����� $x$ ��� ����� �� ����?

% ������ 1 ������� � ���������� ���������� ����(150)
\AddProb ���������� ��������������� ���� ������~$m$, �������~$R$, ������� � ���������� ��������� ���� ��������~$B$. 
����� ����� ���������� ����������� �� ��� ������ � �����~$q$. 
����� ������� �������� ������� ����, ���� ��������� ��������� ����?


\section{III ����}

% ����� ������ 1 ��� (152)
\AddProb ������������� ����� �� ��������� $a$ � $b$ ���������� �� ���������� $L$ �� 
������� ������� � ����� $I$. ����� ������� ������� ����� ��� ���������� ���� � 
�������, ���� �������� ������������� ����� ����� $R$, � �� ���������� 
�������������� ����� ����������?

% ������ 1 � ����� ���������� �������(115)
\AddProb � ������� ������������������ ������ ����� ����������� �������� 
����� $m$ ���������� 1~���� ������������ ���������� ���� ��� �����������~$T_0$. 
� ��������� ������ �������� ������� ���������� � ���� ������� � ����� $v$ � 
$3v$. �� ����� ������������ ����������� ��������� ���? ������ ����� �� 
��������, ������ ���� �� ��������� � ������ ������� ����������.

\AddProb ����������� ���� �������� �� ��� $\gamma$-������ � ��������� $E_1$\,=\,3100~��� � $E_2$\,=\,2000~���. 
����� ���� ������� ������� � ������������ ������� �������, ���� ������� ����� ����� $E_{\pi}$\,=\,135~���.

\section*{Студенческий чемпионат по физике 2007 года}\addcontentsline{toc}{chapter}{Студенческий чемпионат по физике 2007 года}
\section{I ����}

\AddProb � ������� ����� ����� ������� ��������� $k$ �������� �����. ������� ������� 
�� ���������. ����� ����� ���������. ����� ������������ �������� ��������� ����� 
��� ����� ��������? ����� ������ $m$.

\AddProb ���� �������� �� ���� ������ ��� ������� � ������ �������� ������, ���������� 
���� �� �����, �� ������� ������� ����� ������������ � ������. � ����� ��������� 
����� ������������ ��� �������, ���� ���� �� ������� ������� ������������ ������ �� 
��������� 1 ��/�? ����������� ������� ����� 5 ������ � 5 ����������� ����� ���� �� 
���������, � ����������~--~6.

% ������ (82) ������� � �������
\begin{wrapfigure}{r}{4cm}
\includegraphics[width=4cm]{0611EnergyAndImpulseBar.jpg}
\end{wrapfigure}

\AddProb ��� ���� ����� �������� ������ $m$ ������ ��������� �������� ������������ ����� ����� ������ $l$. 
������� �� ���������� ��������� � ������������ ������� ����� �������� � ��������. 
������ ���� �������� ��� ������ �� �������������� �����������, ������� - �� ������������. 
������� �������� �������� ������� ����, ��� ������� ������� ��������� �� ������������ ������.


\section{II ����}

\AddProb ���� �������� �� ���� ������ ��� ������� � ������ �������� ������, ���������� 
���� �� �����, �� ������� ������� ����� ������������ � ������. � ����� ��������� 
����� ������������ ��� �������, ���� ���� �� ������� ������� ������������ ������ �� 
��������� 1 ��/�? ����������� ������� ����� 5 ������ � 5 ����������� ����� ���� �� 
���������, � ����������~--~6.

\AddProb ������� ������� ��������� �������� �������� � ���������� ������������� ����, ������������� �������� $E$\,=\,300~�/��. 
�������� �������� ����� ����������� � ���� ������� ��������� ����� $l\,=\,10^{-8}$~��, �� ������ ������� ��������� ��� ������������ ����� 
����� $m\,=\,10^{-24}$~�, ������� �� ���� ������ $e\,=\,1.6\times10^{-24}$~�� � $-e$ ��������������. 

\AddProb ��� ����������� ��������� ������������ ���� ��� ���������� �������� � ��� 
���������� ������ ������ ����������� ��������� �����. ����������� ���������� ����, 
��������� ����� � �������� �������� ����� �������������� $V_0$ � $p_0$, ��������� 
���������� ����������, ����� ������� � ������� ������������ ������� �������� 
������������� ���: ���� ��� ��� ���������� ������ $V_0$, ������ ��� ��������� �������� $p_1$, 
������ ��� ��� ���������� �������� $p_0$, ������ ����� ���� ���������� ������ $V_1$. ���, 
������� ��� ��������, ���������� ��������� �������������?


\section{III � IV �����}

\AddProb ���-����, ������� ������������ ��������, �������� �� ������������ ������ � ��������� ����� ��������� ����� ������������ ������. 
������� ������ ���� ���������, ������, ��� �� ����� ������ ��������� ������� $\rho$ ������� � ������� ���������� �� �������� 
$\delta\,=\,1.2\times10^{-2}\,\rho$ ����� ������ $h$~=~100~�. ������� ���� � ������ ����������.

\AddProb ��������� ���������� ��� ������ � ����� �������~$A$ ������������� ���� �������������~$E$. �� ���� ������� ����� ����� 
������� ��������. ���� ������ ����� ����� ������������ ���� � �����~$A$?

\AddProb �������, �� ����� ����� �������� ��� ������� ������� $D$~=~0.5~� �� ����� �� ����� �������� ����. ��������, ��� ���������������� ���� 
$\kappa$~=~2.2 ��/(�$\cdot\,$�), �������� ������� �������������� $q\,=\,3.4\times10^5$ ��/��, ��������� $\rho\,=\, 0.9\times10^3$ ��/�$^3$. 
�������, ��� ������� ����������� ��������� � ����� $T_0\approx263$~�. 

\chapter{Студенческий чемпионат по физике 2008 года}
\section{I ����}

\AddProb ��� ������� ������ ������� � ����������� ���������� ���������� �� ����� ����� ����������� �����, ���� ����� $t$ ������ ����� �������. 
��� ����������� � ������� ����� $T$ ������ ����� ������ ������� ������. ���������� ��������� �������� �������. �������������� ������� ����������.

\AddProb ��������� ����� $m_1$ � ������������������ ������� ������� $R$ ����� �� ������� �����. 
���� ����� $m_2$ ������ �� ���� ������ � ���������. ������� �������� ���� � ��������� � 
������, ����� ���� �������� ������ ����� ���������. ������� ����������.

\AddProb �������� ���������� �������� ������� ������� $R$, ����������� ������ ����� 
�������������� ��� �� ��������� $n$ ��/�, ������ � ������������ ��������� �� 
�������������� �����������. ������� �������� $N$ ������� �������, ������ ��� �������� ��� 
��������� �����������? ����������� ������ ���������� ����� ���������� �������� � 
������������, �� ������� �� ���������, �� ������� �� �������� �������� � �����~$\mu$.


\section{II ����}

\AddProb ���������� ������ ���������� �������� ������ $m$ �������������� � �������������� 
��������� ����� ������������� ������� � ������ �������. � ��������� ������ ������� $t$~=~0 ���� ��  
���� ����������. ����� ����, ������� ��������� �� ������ ����� ����� �� ����� ����� �������. 

\AddProb ��� ����������� ��������� �������� ������������� $c_P$ � $c_V$ ���� �������� ������ $T_1$ ����� 
��������� ����� � U-�������� ���������� ������ � ������������ �������. ����� ����� �� ��� ����� 
������ ���� �������� ������� ���������� ����� ���������� ���� � ����������� �����, ���������� 
���� ������ ��������� ��������� � ���� ������ $T_2$. ������ ������� ������ � ���������� ���� � ����� 
��������������, ������� ������� ��� $\gamma = \frac{c_P}{c_V}$. ����� ������� ���� ����� $V$ [��$^3$], �������� ���� � ��� � 
��������� ����� $h$ [�� ��. ��.], � ������� ����������� ������� ������ $S$ [��$^2$]. ������� ������������� 
����� ������ ����� ���������� �� ��������� � ������� ����~$V$.

\AddProb � ������� ����������� �������� $L\times L\times d$ ���������� � ���������� ��������� $v$ �������� 
����������� � ��������������� �������������� $\varepsilon$. ���������� ��� � ���� ������� � ��� $\xi$, 
������������ � ������������.


\section{III ����}

\AddProb ������������ ����� ������ $m$, �� ������� �� ������������ ������� ����, 
��������� ����� ����������� � �������� �������� $R$. ������������� ���� ���� 
�������� �� �������, ��� ��� ������������ ����� �������� ��������. � �����-�� 
������ ������� � ����� $m$ �������� ������� ���� � ���������� ����������� ���, ��� 
���� ������ �������������, ��������� �������� ����� $m$ ��������� $u$. ����� ����� 
��������� ����� ��������� ����.

\AddProb ��������� ����������� ��� �������� ������� ��������� � ����� 1-2-3-1, ������������ ��
�������. ������� 2-3 -- ���� ����������. ������� ��� ������ �����.

\AddProb � ������� ����������� �������� $L\times L\times d$ ���������� � ���������� ��������� $v$ �������� 
����������� � ��������������� �������������� $\varepsilon$. ���������� ��� � ���� ������� � ��� $\xi$, 
������������ � ������������.

\chapter{Студенческий чемпионат по физике 2009 года}
\section{I курс}

\AddProb Два автомобиля движутся к перекрестку по пересекающимся дорогам, образующим 
прямой угол. Скорости автомобилей $v_1$ и $v_2$. Расстояния от автомобилей до перекрестка одинаковы и равны~$L$. 
Каково будет минимальное расстояние между автомобилями в процессе движения?

\AddProb Доска массы $M$ находится на двух одинаковых роликах, вращающихся навстречу друг другу. Расстояние между осями вращения роликов равно~$L$. 
Коэффициент трения между доской и роликом равен~$\mu$. Найдите период колебаний доски.

\AddProb Обруч радиусом $r$ свободно падает в поле силы тяжести, вращаясь вокруг собственной оси, расположенной в горизонтальной плоскости, 
с угловой скоростью $\omega$\,($\omega^2r > g$). Определите радиус кривизны траектории точки обруча, занимающей в данный момент наинизшее положение, 
если скорость падения обруча равна~$v_0$.


\section{II курс}

\AddProb Вася спешит на свидание с Машей, ожидающей его в беседке, находящейся на расстоянии $h$ от аллеи, по которой бежит Вася. 
По какому пути должен двигаться Вася, чтобы достичь желанной цели в кратчайшее время, если его скорость по аллее $v_1$, а по саду $v_2$ ($v_1>v_2$).

\AddProb Металлическая рамка расположена вертикально в однородном магнитном поле, вектор индукции $\vec B$ которого направлен перпендикулярно плоскости рамки. 
Вертикально вниз из состояния покоя начинает движение металлический стержень $AC$. Стержень находится в электрическом контакте с 
вертикальными сторонами рамки, но движется без трения. Определите скорость движения стержня через время $t$ после начала движения. 
Масса стержня $m$, электрическое сопротивление $R$, расстояние между контактами $L$. Электрическим сопротивлением и индуктивностью рамки пренебречь.

\AddProb Небольшое тело скользит по вогнутой поверхности, имеющей параболическую форму (уравнение поверхности $y = ax^2$, где $a$ -- константа). 
Коэффициент трения между телом и поверхностью $\mu\ll 1$. В начальный момент времени координата тела $x_0 = 2.75\mu/a$, а скорость нулевая. 
Какова будет координата тела, когда оно окончательно остановится?


\section{III курс}

\AddProb Внутри гладкой сферы радиуса $R$ находится маленький шарик массы $m$ с зарядом $+q$. Какой заряд $Q$ нужно поместить в нижней точке сферы, 
чтобы шарик удерживался в верхней точке? Поляризацией сферы можно пренебречь.

\AddProb Найти время исчезновения мыльного пузыря радиуса $R_0$, соединенного с атмосферой капилляром, который имеет длину $L$ и радиус канала $r$. 
Поверхностное натяжение $\sigma$, вязкость воздуха~$\eta$.

\AddProb Частица с зарядом $q$ и массой $m$ движется с начальной скоростью $v_0$ в вязкой среде в поперечном магнитном поле с индукцией $\vec B$. 
Сила вязкого трения равна $\vec F = -r\vec v$, где $r$ -- константа. На каком расстоянии от начальной точки частица остановится?

\chapter{Студенческий чемпионат по физике 2010 года}
\section{I курс}

\AddProb В U-образной трубке скользит упругая нерастяжимая нить. Скорость входящего в трубку конца равна $v$, выходящего –- $u$. 
С какой скорость движется трубка?

\AddProb Брусок массы $m$ тянут за нить так, что он движется с постоянной скоростью по горизонтальной плоскости с коэффициентом трения $\mu$. 
Найти угол $\alpha$, при котором натяжение нити минимально. Чему оно равно?

\AddProb Однородный шар радиуса $r$ скатывается без скольжения с вершины сферы радиуса $R$. Найти угловую скорость шара после отрыва от сферы. 
Начальная скорость шара пренебрежимо мала.


\section{II курс}

\AddProb На гладком столе лежит тонкое кольцо массы $M_1$ и радиуса $R$. На него сверху кладут шероховатое кольцо такого же радиуса, 
которое вращается с угловой скоростью $\omega$. Масса верхнего кольца $M_2$. Пренебрегая трением нижнего кольца о стол, определите, 
какая угловая скорость вращения колец на столе установится через большой промежуток времени. Сколько тепла выделится при установлении вращения?

\AddProb Тонкая трубка изогнута в виде почти замкнутой окружности. В этой трубке скользит упругая нерастяжимая нить. 
Скорость участка нити, входящего в трубу, равна $v$, скорость трубы равна $2v$. Обе скорости одинаково направлены. 
Какой участок нити в трубе движется с максимальной скоростью? Чему равна эта скорость?

\AddProb Металлический стержень массой $m$ и длиной $l$ подвешен на двух легких проводах длиной $R$ в магнитном поле с индукцией $B$, 
вектор которой направлен вертикально. К точкам крепления проводов подключен конденсатор емкостью $C$, заряженный до напряжения $U$. 
Сопротивление стержня и проводов пренебрежимо мало. Найти максимальный угол отклонения проводов от вертикали, 
если разрядка конденсатора происходит за очень малое время.


\section{III курс}

\AddProb Тонкая трубка изогнута в виде почти замкнутой окружности. В этой трубке скользит упругая нерастяжимая нить. 
Скорость участка нити, входящего в трубу, равна $v$, скорость трубы равна $2v$. Обе скорости одинаково направлены. 
Какой участок нити в трубе движется с максимальной скоростью? Чему равна эта скорость?

\AddProb По двум вертикальным рейкам, соединенным вверху и внизу сопротивлениями $R$, может скользить без трения проводник, 
длина которого $L$, масса $m$, сопротивление $R$. Система находится в однородном магнитном поле, индукция которого $B$ перпендикулярна плоскости рисунка. 
Найти максимальную скорость проводника в поле силы тяжести, если пренебречь сопротивлением реек.

\AddProb Теплоизолированный цилиндрический сосуд разделен теплоизолирующим поршнем на две равные части объемом $V$ каждая. 
Давление одноатомного газа в левой половине сосуда $p$, в правой $2p$, а температура одна и та же и равна $T$. Систему предоставили самой себе. 
Определите конечное давление в обеих частях сосуда после завершения переходных процессов, если поршень перемещается внутри сосуда без трения.

\chapter{Студенческий чемпионат по физике 2011 года}
\section{I курс}

\AddProb Два тела бросили одновременно из одной точки: одно вертикально вверх, другое под углом $\varphi~=~30^{\circ}$ к горизонту. 
Начальная скорость каждого тела $v$~=~10 м/с. Найти расстояние между телами через $t$~=~1~с.

\AddProb Тонкий стержень массы $m$ и длины $l$ лежит на гладкой горизонтальной поверхности. Пластилиновый шарик массы $m$ со скоростью $v$, 
перпендикулярной к стрежню, ударяется об один из его концов и прилипает к нему. Какое количество теплоты выделится при таком ударе?

\begin{wrapfigure}{r}{3cm}
\includegraphics[scale=0.5]{1309OscillationsDisc.jpg}
\end{wrapfigure}

% Почти задача (179) Колебания
\AddProb Тонкий однородный диск массы $m$ и радиуса $R$, подвешенный в горизонтальном положении к упругой нити, совершает крутильные колебания в жидкости. 
Момент упругих сил со стороны нити $M~=~\alpha\,\varphi$.  Сила сопротивления на единицу поверхности $F~=~\beta\,v$, где $v$ -- локальная скорость 
движения диска. Найти частоту малых колебаний.


\section{II и III курсы}

\AddProb Магнитная лента с катушки протягивается через звукосниматель с постоянной скоростью $v$. Толщина ленты равна $h$. 
Найти угловую скорость катушки как функцию времени $t$, если в момент времени $t$~=~0 радиус внешнего слоя магнитной ленты равен $R$.

\AddProb С какой силой давит на землю кобра, когда она, готовясь к прыжку, поднимается вертикально вверх с постоянной скоростью $v$? 
Масса змеи $m$, ее длина $l$.

\AddProb Катушка состоит из среднего цилиндра радиусом $r$ и двух крайних цилиндров радиусами $R > r$. Длинный тонкий провод плотно наматывают на катушку 
следующим образом: сначала обматывают один из крайних цилиндров, а затем продолжают наматывать этот же провод на средний цилиндр в том же 
направлении, в каком начинали намотку. После завершения намотки катушку кладут на горизонтальный стол, помещённый в однородное постоянное магнитное поле $B$, 
линии индукции которого параллельны оси катушки. К первому концу провода, лежащему на столе, подсоединяют идеальный вольтметр, а другой конец провода, 
касающийся неподвижного скользящего контакта, соединённого с вольтметром, начинают тянуть вдоль поверхности стола с постоянной скоростью $v$ в направлении, 
перпендикулярном оси катушки. Считая, что катушка катится по столу без проскальзывания, найдите показания вольтметра.

\chapter{Студенческий чемпионат по физике 2012 года}
\section{I курс}

\AddProb Однородный стержень массой $m$ и длиной $L$ вращают в горизонтальной плоскости с угловой скоростью $\omega$ вокруг одного из его концов. 
Найдите зависимость натяжения стержня от расстояния $x$ до оси вращении, если на другом конце закреплен маленький грузик массой $M$.

\AddProb Газовый термометр состоит из двух одинаковых сосудов  вместимости $V_0$ каждый, соединенных трубкой длины $L$ и сечения $S$. 
Трубку перекрывает капля ртути. Сосуды наполнены газом. Если температура газа в обоих сосудах одинакова, ртуть находится посередине трубки. 
Один сосуд помещен в термостат с температурой $T_0$. Проградуируйте термометр, находя зависимость температуры газа во втором сосуде 
от смещения ртути из положения равновесия.

\AddProb Однородный цилиндр радиуса $R$ и массы $m$ толкнули с начальной скоростью $v_0$ без вращения вдоль горизонтальной плоскости. 
Через какое время прекратится проскальзывание, если коэффициент трения цилиндра о плоскость равен $\mu$? Какая часть начальной энергии перейдет в тепло?


\section{II и III курсы}

\AddProb Газовый термометр состоит из двух одинаковых сосудов  вместимости $V_0$ каждый, соединенных трубкой длины $L$ и сечения $S$. 
Трубку перекрывает капля ртути. Сосуды наполнены газом. Если температура газа в обоих сосудах одинакова, ртуть находится посередине трубки. 
Один сосуд помещен в термостат с температурой $T_0$. Проградуируйте термометр, найдя зависимость температуры газа во втором сосуде 
от смещения ртути из положения равновесия.

% Почти задача (185) Колебания
\AddProb Проволочная вешалка качается с малой амплитудой в плоскости чертежа относительно заданных положений равновесия. 
В положениях а и б длинная сторона вешалки расположена горизонтально. Две другие стороны равны между собой. 
Во всех трех случаях (а, б, в) возникают колебания с одинаковыми периодами. Где расположен центр масс вешалки, если распределение массы в деталях неизвестно?

\begin{figure}[!h]
\includegraphics[scale=0.35]{1315OscillationsHanger.jpg}
\end{figure}

\AddProb Грани правильного додекаэдра равномерно заряжены с одинаковой поверхностной плотностью $\sigma$. 
В центре додэкаэдра помещен точечный заряд $ $. Определить силу, с которой точечный заряд действует на одну грань. Взаимодействие граней не учитывать. 

\chapter{Студенческий чемпионат по физике 2013 года}
\section{I курс}

\AddProb Пассажир первого вагона поезда длины $l$ прогуливался по перрону. Когда он был рядом с последним вагоном, поезд начал двигаться с ускорением $a$. 
Пассажир сразу же побежал со скоростью $v$. Через какое время он догонит вагон?

\AddProb Частица массы $m$ влетает в область, где на нее действует тормозящая сила, пропорциональная расстоянию между частицей и границей области. 
Найдите эту зависимость, если глубина проникновения частицы в область торможения пропорциональна ее начальному импульсу: $l~=~\alpha\,p$.

\AddProb Небольшую шайбу $A$ положили на наклонную плоскость, составляющую угол $\alpha$ с горизонтом, и сообщили начальную скорость $v_0$. 
Найти зависимость скорости шайбы от угла $\varphi$, если коэффициент трения $k~=~ \tg \alpha$ и в начальный момент $\varphi_0~=~\pi/2$.


\section{II курс}

\AddProb Пассажир первого вагона поезда длины $l$ прогуливался по перрону. Когда он был рядом с последним вагоном, поезд начал двигаться с ускорением $a$. 
Пассажир сразу же побежал со скоростью $v$. Через какое время он догонит вагон?

\AddProb Два одинаковых шарика, имеющие одинаковые заряды $q$, соединены пружиной. Шарики колеблются так, 
что расстояние между ними меняется от $l$ до $4l$. Найти жесткость пружины, если ее длина в свободном состоянии равна~$2l$.

\AddProb Замкнутая стеклянная трубка с отводом, погруженным в открытый сверху сосуд с ртутью, в верхней своей части содержит столбик воздуха. 
Его границы с ртутью находятся на расстоянии $R$ от оси симметрии системы. Определите, с какой угловой скоростью нужно вращать систему вокруг этой оси, 
чтобы давление воздуха изменилось в $n$ раз. Начальное давление воздуха $p_0$, плотность ртути $\rho$, ее уровень в сосуде можно считать неизменным.

\chapter{Студенческий чемпионат по физике 2014 года}
\section{I ����}

\AddProb ��� ����� ����� � ��������� ������� ������� ������ � ������� ���� � ������� $ 45^{\circ}$, 
����� �� ���� �� ����� �� ������������ ���������� �� ����� ������?

\AddProb �� ���������� ����������� ���������� ������� � ����� $2\alpha$ ��� ������� �� ������ $h$ �� ������� ��������� ����� ����. 
����������� ������ ����� ����� � ������������ ������� ����� $\mu$. 
����� ����������� ������� �������� �������� ������ ������ ������������ ��� $\omega$, ��� ������� ���� ����� ���������� � �������.

\AddProb ���� ������������ ���������� ���� ��������� � �������� ��� �������, ������������ � ��������� ���������� ��������, ������������� ������ ����. 
������ �������� � ������� ��������������, � ��� �������� �����. ��������� ����� ���� $V_0$, ��� ������� ������� �� �������������, �������� ���, 
��� $p_0\,S^2\,=\,k\,V_0$, ��� $p_0$ -- �������� ����������� ��������, $S$ -- ������� ������, $k$ -- ����������� ��������� �������. 
����� ������������ ���� ��� ����� ��������.


\section{II � III �����}

\AddProb ��� ����� ����� � ��������� ������� ������� ������ � ������� ���� � ������� $ 45^{\circ}$, 
����� �� ���� �� ����� �� ������������ ���������� �� ����� ������?

\AddProb ��� ���������� ������ ������� ���������� ����� � ������ ����� $m_1$ � $m_2$ ����� �� ������� ����������� ����������� ���� �����. 
������� ����� ��������� ��� ������ ������ ������������ ����������� ����, ��������� ���� �� ����� �� ��������� ����� ��������. 
��������� ����� �������� ��������� ��������� �������� ��������� $k$. ����� ������ ����� ��������� �������.

\AddProb �� ������� ���������� ����� ����������������� ������ ��� ��������� �������������� �������. 
������� ����� � ������������� �������� ���������� � ��������� ����, ���������������� ��� �����; 
����� ���� �� ������� � ���������������� (� ���������� ���� � ��� �����) ����������� ������������ ���. 
����� ���������� �������� $\Delta P$, ����������� �������. ��������, ������������ �� �������, ������� ����������.

\chapter{Студенческий чемпионат по физике 2015 года}
\section{I курс}

\AddProb Утка летела по горизонтальной прямой с постоянной скоростью $u$. 
В неё бросил камень неопытный охотник, причём бросок был сделан без упреждения, 
т. е. в момент броска скорость камня $v$ была направлена точно на утку под углом $\alpha$ к горизонту. 
На какой высоте летела утка, если камень всё же попал в неё?

\AddProb Катушку ниток радиусом $R$ пытаются, прислонив к стене, удержать на весу с помощью собственной нитки, отмотанной на длину $l$. 
При каких значениях коэффициента трения между катушкой и стеной это возможно?

\AddProb Тепловая машина с идеальным газом в качестве рабочего вещества совершает обратимый цикл, 
состоящий из изохоры 1--2, адиабаты 2--3 и изотермы 3--1. 
Найти КПД машины как функцию максимальной $T_2$ и минимальной $T_1$ температур, достигаемых газом в этом цикле.


\section{II и III курсы}

\AddProb Материальная точка массой $m$ начинает двигаться со скорость $v_0$ в вязкой среде, в которой на тело действует сила аэродинамического сопротивления, 
пропорциональная квадрату скорости: $F~=~-\beta\,v^2$, где $\beta$ -- постоянный коэффициент сопротивления, зависящий от формы тела. 
Сила гравитационного притяжения на тело не действует. Найдите зависимость пройденного пути от времени. 
Может ли тело остановиться, если описывать его движение в рамках данной модели?

\AddProb Однородный цилиндрический блок массы $M$ и радиуса $R$ может свободно поворачиваться вокруг горизонтальной оси $O$. 
На блок плотно намотана нить, к свешивающемуся концу которой прикреплен груз $A$. 
Этот груз уравновешивает точечное тело массы $m$, укрепленное на ободе блока, при определенном значении угла $\alpha$. Найти частоту малых колебаний системы.

\AddProb При каком условии амплитуда тока $I$ в цепи, изображенной на рисунке, зависит только от амплитуды приложенного напряжения 
$U~=~U_0\,\cos\,\omega\,t$, но не от его частоты? Найти при этом условии разность фаз между приложенным напряжением и напряжением на концах $RC$-пары.

\part{Олимпиада "Юные таланты"}

\chapter{Олимпиада "Юные таланты" 2008 года}
\section{10 класс}

\AddProb Найти сопротивление цепи между точками $A$ и $B$. Сопротивление каждого резистора известно и равно~$R$.

\AddProb Легкий жесткий стержень с шариком массы $m$ на конце свободно вращается в вертикальной плоскости вокруг точки~$O$. 
Известно, что в верхней точке траектории модуль силы натяжения стержня равен $T$ и в два раза меньше, чем в нижней. 
Найдите отношение скоростей шарика в верхней и нижней точках траектории. Ускорение свободного падения равно~$g$.

\AddProb С балкона, находящегося на высоте 20 м, бросают мяч со скоростью 20~м/с. 
Мяч упруго ударяется о стену соседнего дома и падает на землю под балконом. Определите расстояние до соседнего дома, если время полета мяча 1,4~с.

\AddProb В цилиндрическом стакане с водой плавает брусок высоты $L$ и сечения~$S_1$. 
При помощи тонкой спицы брусок медленно опускают на дно стакана. Какая работа при этом совершена? 
Сечение стакана $S_2\,=\,2S_1$, начальная высота воды в стакане тоже $L$, плотность материала бруска $\rho\,=\,0.5\rho_B$, , где $\rho_B$ -- 
известная плотность воды.



\section{11 класс}

\AddProb К концу вертикально висящей пружины длины $l$ прикрепили груз, в результате чего ее длина возросла до~$2l$. 
Предполагая, что удлинение пружины пропорционально нагрузке, найти угловую скорость груза, 
вращающегося на этой пружине по кругу в горизонтальной плоскости, если длина пружины в этом случае~$L$. Массой пружины пренебречь.

\AddProb На горизонтальной поверхности стоят два одинаковых кубика массой~$M$. 
Между кубиками вводится тяжелый клин массой $m$ с углом при вершине~$2\alpha$. Чему равны ускорения кубиков? Трением пренебречь.

\AddProb В плоский конденсатор вдвигается с постоянной скоростью $v$ пластина из диэлектрика. 
Определите ток в цепи батареи, подключенной к конденсатору. Считать известным ЭДС батареи $E$, диэлектрическую проницаемость {\Large $\varepsilon$}, 
высоту пластины $h$, площадь квадратных пластин конденсатора~$S\,=\,b^2$.

\AddProb 4.	Одноатомный газ участвует в некотором процессе. Известно, что его внутренняя энергия пропорциональна квадрату объема. 
Найдите работу, которую совершает газ при сообщении ему количества теплоты~$Q$.


\chapter{Олимпиада "Юные таланты" 2009 года}
\section{10 �����}

\AddProb ������ �������� ����������� ������ � $3$ ���� ������ ������� ����, �� ������� ������������ ����. 
������ �������� �������� ����� ��������, ���� ����� � ������� $10$~� ����������� �� $20$~�.

\AddProb �������, �� ����� ����������� ����� ������� ������ ������ �������� ���������� ���� ������� 500 �$^3$, 
����� �� ��� ������� �������� ������ 70~��? ����� �������� ���� ����� 30~��.

\AddProb ���� ������ $m$ ������ �� ����� ��������� ����� � ������ $H$ � ���������� ��������� ������� ����. 
�� ����� �������� $h$ �������� ������� ����� ����� �����, ���� ����� ����� ����� $M$, � ����������� ��������� ������� ����� $k$?

%������ 118, �����������, ���
\begin{wrapfigure}{r}{3cm}
\includegraphics[scale=0.25]{0904LawOfConservationOfEnergyEfficiency.jpg}
\end{wrapfigure}

\AddProb ������� ��� �������� ������, ���� ������� ������� �� ���� ������ � ���� ������, � ������� ����� �������� ��������� ����������� ���. 
�������� ������ ������� � ����� ������� ����� �� ��������, ��������������� ����������� $T_1$, 
� �������� ������� ������� � ������ ������� -- �� ��������, ��������������� ����������� $T_2$.



\section{11 �����}

\AddProb �� �������������� �������������� ����, �������� � ���������� ����� �������� �� �������� ������. 
� ����� ��������� $v$ ������ ��������� ����� ��������, ���� ����� ��������� $q_1\,=\,-1.6\cdot~10^{-19}$~��, 
����� ���� $q_2\,=\,1.6\cdot~10^{-19}$~��, ������ ������ ����� ������� ������ $r\,=\,0.5\cdot 10^{-10}$~�, ����� ��������� $m\,=\,9.1\cdot 10^{-31}$~��.

\AddProb ����, ������� ����� ���� � ������� ����� $L$, ��������� �� �������������� �����������. 
��� ����� ��������� ������������ ������ ����� ������ � ������������ ������ ��� ����������� ����� ������� �� ���������� $L$ ����� ������, 
��� ������ ��� ���������� (������������� ����� �����)?

% ������ 159, �������������, ���
\begin{wrapfigure}{r}{5cm}
\includegraphics[scale=0.5]{1214EMIRailsAndBridge.jpg}
\end{wrapfigure}

\AddProb ������������ ������ ������ $2L$ ���������� �� �������������� ��������� �� ���������� $l$ ���� �� �����. 
� �� ������ ������������ ��� ���������� ������� � ��� {\Large $\varepsilon$}. �� ������� ����� ��������� ����� $m$, 
������� ����� ������������� ��������� ����� ���. ��� ������� �������� � ���������� ������������ ��������� ���� � ��������� $B$. 
������, ��� ������������� ��������� ����� $R$, � ������������� ������� ����� ������� �� ������� ����� $\rho$, ������� ������ ����� ���������, 
����������� ��� �������� ��������� �� ��������� ����������, ����������� ����������, ���������� �������������� ����������, 
�������������� ���������, � ����� �������������� ����.

\AddProb �������������� �������� ������������������ ������� �������� �� ��� ����� ������ ��������������� �������, 
������� ���������� �������� � ����� �� �������� ������ ��������. ����� � ������ �� ������ ��������� �� $\nu$ ����� ���������� ������������ ����. 
��������� ����������� ������� $T$, ����� �������� $2L$, ����������� ����� ������� $L/2$, ��������� ������� � ��������� ���������� ����� $L/4$. 
� ������ ��������� ���������. ��� ��������� ����������� ���� ������� ����� ������������ ������ ����������? 
�������������� ��������, ������ � ������� ����������, ������ ���.

\chapter{Олимпиада "Юные таланты" 2010 года}
\section{10 класс}

\AddProb 

\AddProb 

\AddProb 

\AddProb 



\section{11 класс}

\AddProb 

\AddProb 

\AddProb 

\AddProb 

\chapter{Олимпиада "Юные таланты" 2011 года}
\section{10 класс}

\AddProb Мимо остановки по прямой улице проезжает грузовик со скоростью 10~м/с. Через 5 с от остановки вдогонку грузовику отъезжает мотоциклист, 
движущийся с ускорением 3~м/$c{^2}$. На каком расстоянии от оснановки мотоциклист догонит грузовик?

\AddProb На гладкий стол помещен брусок массой 1 кг, на котором лежит коробок массой 50~г. Брусок прикреплен к одному из концов невесомой пружины, 
другой конец которой закреплен в неподвижной стенке. Брусок отводят от положения равновесия перпендикулярно стенке на расстояние $L$ 
и отпускают с нулевой начальной скоростью. При каком значении $L$ коробок начнет скользить по бруску? 
Коэффициент трения бруска о брусок 0.2, жесткость пружины~500~Н/м. Трением бруска о стол пренебречь.

\AddProb Шарик массой 100~г подвешен на нити длиной 1~м. Его приводят в движение так, что он вращается по окружности, лежащей в горизонтальной плоскости, 
которая находится на расстоянии 1/2~м от точки подвеса. Какую работу $A$ нужно совершить для реализации такого движения?

\AddProb При движении трамвая по горизонтальному участку пути с некоторой скоростью его двигатель потребляет ток 100 А. КПД двигателя 0.9. 
При движении трамвая по наклонному участку пути вниз с той же скростью двигатель тока не потребляет. Какой ток будет потреблять двигатель при 
движении трамвая по тому же участку пути вверх с той же скоростью? При решении задачи учесть, что КПД двигателя зависит от потребляемого тока.



\section{11 класс}

\AddProb В электрической цепи, показанной на рисунке, ЭДС источника тока равна 12 В, емкость конденсатора $C\,=\,2$~мФ, 
индуктивность катушки $L\,=\,5$~мГн, сопростивление лампы $R\,=\,5$~Ом и сопротивление резистора $r\,=\,3$~Ом. 
Какая энергия выделится в лампе после размыкания ключа? Сопротивлением катушки и проводов пренебречь.

\AddProb Преследуя добычу, гепард движется по прямой горизонтальной тропе прыжками длиной 8 м. Внезапно на пути гепарда встречается овраг глубиной 4/3 м. 
Отталкиваясь от края оврага точно так же, как и при движени по тропе, гепард прыгает в овраг. Найти горизонтальное перемещение гепарда при этом прыжке, 
если горизонтальная составляющая его скорости 108 км/ч. Сопротивление воздуха не учитывать, дно оврага считать горизонтальным.

\AddProb В цилиндре под подвижным поршнем находится идеальный газ, поддерживаемый при постоянной температуре. 
Когда на поршень положили груз массой 1 кг, объем газа уменьшился в 2 раза. Какой массы груз нужно положить на поршень дополнительно, 
чтобы объем газа уменьшился еще в 3 раза?

\AddProb Небольшое тело скользит по вогнутой поверхности, имеющей параболическую форму (уравнение поверхности $y=ax^2$, где а -- константа). 
Коэффициент трения между телом и поверхностью $\mu \ll 1$. В начальный момент времени координата тела $x_0\,=\,2.75\,\mu/a$, а скорость нулевая. 
Какова будет координата тела, когда оно окончательно остановится?

\chapter{Олимпиада "Юные таланты" 2012 года}
\section{8 �����}

\AddProb ��� �����: ������ ����������� ����� ������ ����, ���� ������� �� ����� �����? ������?

\AddProb ���� �������, ��� ���� ������� � ������ ��������, ������������ �� 2/3 ������ ������. 
���� ��� ��������� � ����, �� ��� �������, ������������ ����������. ���� ����� ��������� ������ ��������, ���� ��������� ���� 1~�/��$^3$?

\AddProb ������, ��������� � ������ �����, ���������� � �������� ������� �������� ������ ����, � ����� ����� � �������, �������� � ������������ ���������. 
����� ����������� � �������� ����� ������ � ������ ������ �� $\Delta L = 5$~�� ������, ��� �� ������. 
������� ������ $H$ ���� ��������, �������� � ������ � ������ ������. ��������� ����������� ������� ������ $S_1$ � �������� $S_2$ �����~0.5.

\AddProb �� ������ ������ ��������� ������������, ����������� � ������� ����� ����. 
� ��������� ������ ������� ���������� �� �������� �� ������������� � 2 ���� ������, ��� �� ������������. 
������������ � ����������� �������� ��������� ��������� ���� ����� �� ���������� 20~��/� � 60~��/� ��������������. 
� ����� ������� � � ����� ��������� ������ ���� �������, ����� ������������ ����������� � �������������� � ������������� � ����� �� �������?



\section{9 �����}

\AddProb ���� ���������� $\rho$ ������� �� ������� ������� ���� ��������� � ����������� $\rho_1$ � $\rho_2$. 
������� ��������� ������ ����, ������������ � ������ �������� $V_1$ � ������, ������������ � ������� ��������~$V_2$.

\AddProb �� ������ ������ ��������� ������������, ����������� � ������� ����� ����. 
� ��������� ������ ������� ���������� �� �������� �� ������������� � 2 ���� ������, ��� �� ������������. 
������������ � ����������� �������� ��������� ��������� ���� ����� �� ���������� 20~��/� � 60~��/� ��������������. 
� ����� ������� � � ����� ��������� ������ ����  �������, ����� ����������� � �������������� � ������������� � ����� �� �������?

\AddProb � ������ ������ �������� $V = 2$~� ������� ����� ��� ����������� $T_1 = 78$~�. �� ����� �������� ����� ���������� ����������. 
���������� �������� ������� ��������� �����, ���� ��������, ��� 40~� ���� ��� ����������� $T_2  = 273$~� � ��� ������ ������� � ������� 22.5 �����. 
�������� ������� ����� ������ ������ ������� ���������������� �������� ���������� ������ � ������� ������ ������. 
����������� ����������� ������� $T  = 293$~�, ��������� ������� ����� ��� 78~� ���������� 800~��/�$^3$. 
�������� ������� ��������� ���� $\lambda\,=\,334\cdot 10^3$~��/��.

\AddProb ���������� ���� ���� � ���������� $AB$ � ������������� ����, ���������� �� �������. 
���������� � ���� $U$ � ������������� ������� �� 10 ���������� ���������� $R$ �������� ����������.



\section{10 �����}

\AddProb �������� �� ������ �����, ������� ������ ������ ������������� �� ��������� $v_1\,=\,5$~�/�. 
������ ����� �������� $v_2$ ����� ������������ ��������, ���� �� ������ ������ �������������, �������� ��� ������ ������� ������, 
�� ���� �� ������� ����? ����� ����� $m\,=\,1$~��, ����� �������� $M\,=\,50$~��. ������� � ��� ����������.

\AddProb �� ����� �������� ��������� ������, ������� �� ���� �� ��������� $v\,=\,20$~��/�, �������, ��� ��� � �������� ��� �������� $N_1\,=\,50$ �������� 
� ������. �������� �������, �� ���������, ��� � ��� �������� $N_2\,=\,30$ �������� � ������. 
������� ������������ �������� (���������� �������� � ������� ������) � ��������� ������ ��������� � ��������, 
���� ������� ��� ���������� $S\,=\,24$~��$^2$, � ������ �������� $l\,=\,1$~��.

\AddProb � ������� ������������������ ����� ����� ����� ����������� �������� ����� $m$ ������ ��������� 1 ���� ������������ ���� ��� ����������� $T_0$. 
� ��������� ������ �������� ������� ���������� � ���� ������� � ����� $3v$ � $v$. �� ����� ������������ ����������� ��������� ���? 
������ ����� �� ��������. ������ ���� �� ��������� � ������ ������� ����������.

\AddProb ������� ������ ������������� ����������� ������������� ����, ������������ �� �������. �������� $R$ ������� ���������.



\section{11 �����}

\AddProb �� ����� ����������� ���������� $l$ �� ����������� ������ �������� ��������� ����� ��� ������� ����� ���������, 
���� ���������� �������� �� ��������� $v\,=\,100$~��/���, � ����������� ������ ����� ������ � ������� �����~0.4?

\AddProb ������������������ ������� �������� ��������� ��������������� ������� �� ��� �����. � ����� ����� �������� ��������� �����, � � ������ - �����. 
� ��������� ������ ����������� ����� ����� 300~�, � ������ -- 900~�, � ������, ���������� ������, ���������. 
�� ������� ��� ��������� �����, ���������� ������, ����� ������������ ��������� ����������, ���� ������� ������������ ��� ������? 
������������� �������� � ������ ����������. �������� ����� ������ 40~�/����, �������� ����� ����� 4~�/����.

\AddProb �� ������ ����������� �������������� �� �������� 1~� ������������ ������� �� ����������� � ������ ����� ���������� ���������� ������. 
��������� �������� ������ 1~�/�. ����� ���������� ����� ��������� ����������� � ��� ����, �� �������� ������ ����� ����������� �����. 
�� ������ ������������ ���������� ��������� �������?

\AddProb ���� ������ ������� $U$-�������� ������ � ������� ���������� �������� �������� $S$ �����������, 
� ������ ��������� � ��������� ��� �����~$\alpha$. � ������ ������ �������� ���������� $\rho$ � ������ $M$ ���, ��� � ������� � ��������� ������ ����, 
��� � ������������, ������� ������� ����� �������, ����������� � ������������ �������� ���������~$k$. ������� ������ ����� ��������� ���� �������. 
��������� ���������� ������� �����~$g$.

\chapter{Олимпиада "Юные таланты" 2013 года}
\section{8 класс}

\AddProb Зачем в погребах в холодную погоду рядом с овощами ставят большие емкости с водой?

\AddProb В 100 г воды при температуре $10^{\circ}C$ опущено 40 г льда, имеющего температуру $-10^{\circ}C$. При каком соотношении воды и льда 
возникнет состояние теплового равновесия в этой системе, если она теплоизолирована? Удельная теплоемкость льда 2,5~кДж/кг.

\AddProb Два велосипедиста и пешеход одновременно отправились из пункта $A$ в пункт $B$. 
Более чем через час после выезда у первого велосипедиста сломался велосипед, и он продолжал путь пешком, двигаясь в 4,5 раза медленнее, чем на велосипеде. 
Его обгоняют: второй велосипедист –- через 5/8 ч после поломки, а пешеход –- через 10,8 ч после поломки. 
К моменту поломки второй велосипедист проехал расстояние в два раза большее, чем то, которое прошел пешеход к моменту, на 5/36 ч более позднему, 
чем момент поломки. Через сколько часов после начала движения сломался велосипед?

\AddProb Длинная вертикальная трубка погружена одним концом в сосуд с ртутью. В трубку наливают $m\,=\,0.71$~кг воды, которая не вытекает из трубки. 
Определите изменение уровня ртути в сосуде. Диаметр сосуда $D\,=\,0.06$~м, плотность ртути $\rho\,=\,13600$~кг/м$^3$. 
Толщиной стенок трубки можно пренебречь.

\AddProb Легкая цилиндрическая палочка длиной $L\,=\,20$~см и плотностью $\rho_1\,=\,800$~кг/м$^3$ погружена вертикально в жидкость плотностью 
$\rho_2\,=\,1000$~кг/м$^3$. Нижний конец палочки находится на глубине $H_0\,=\,1$~м. На какую высоту $h$ выпрыгнет палочка из жидкости, 
если ей дать возможность двигаться? Вязкостью жидкости пренебречь.



\section{9 класс}

\AddProb Найдите сопротивление «звезды» между точками $A$ и $B$, если сопротивление каждого звена равно~$R$.

\AddProb Длинная вертикальная трубка погружена одним концом в сосуд с ртутью. В трубку наливают $m\,=\,0.71$~кг воды, которая не вытекает из трубки. 
Определите изменение уровня ртути в сосуде. Диаметр сосуда $D\,=\,0.06$~м, плотность ртути $\rho\,=\,13600$~кг/м$^3$. 
Толщиной стенок трубки можно пренебречь.

\AddProb Железный шарик радусом 1 см, нагретый до $20^{\circ}C$, положен на лед. На какую глубину погрузится шарик в лед, если удельная теплоемкость железа 
$c_1\,=\,475$~Дж/кг$^{\circ}C$. Плотность льда $\rho_2\,=\,900$~кг/м$^3$, плотность железа $\rho_1\,=\,7900$~кг/м$^3$. 
Температура льда $0^{\circ}C$, удельная теплота плавления льда $\lambda\,=\,334$~кДж/кг. Теплопроводностью льда и нагревом воды пренебречь.

\AddProb Найдите скорость верхней точки пересечения двух катящихся колес (точка $A$ на рисунке) в тот момент, 
когда она находится на одной горизонтали с центром большого колеса. Скорость колес одинаковы и равны $v$, радиусы колес $r$ и~$R$.



\section{10 класс}

\AddProb Как изменится сопротивление цепи, состоящей из пяти одинаковых проводников сопротивлением R каждый, 
если добавить еще два таких же проводника, как показано штриховыми линиями на рисунке?

\AddProb На столе в один ряд лежат 10 кубиков. С какой силой нужно, взявшись за два крайних руками, сдавить кубики, чтобы оторвать их от стола? 
Массы кубиков $m$, коэффициент трения кубика о кубик~$\mu$.

\AddProb С одним молем идеального газа совершают цикл, который на PV-диаграмме изображается окружностью. Найдите максимальную температуру газа.

\AddProb На вбитом в стену гвозде на нити длиной $L$ висит маленький шарик. 
Под этим гвоздём на одной вертикали с ним на расстоянии $l < L$ вбит второй гвоздь. 
Шарик отводят вдоль стены так, что нить принимает горизонтальное положение, и отпускают без толчка. 
Найдите расстояния $l$, при которых шарик перелетит через нижний гвоздь. Нить невесома и нерастяжима, трения нет.



\section{11 класс}

\AddProb В цилиндре, закрытом подвижным поршнем, находится газ, который может просачиваться в зазор вокруг поршня. 
В опыте по изотермическому сжатию газа его объем уменьшился вдвое, а давление газа упало в 3~раза. 
Во сколько раз изменилась внутренняя энергия газа в цилиндре? (Газ считать идеальным.)

\AddProb На абсолютно гладкой проводящей поверхности удерживают два металлических кубика. 
На один кубик помещен положительный заряд~$+q$, на другой -- отрицательный~$-q$. Кубики отпускают. Через какое время они столкнутся? 
Расстояние между кубиками $h$ много меньше стороны кубика~$a$. Плотность материала~$\rho$. 

\AddProb На вбитом в стену гвозде на нити длиной $L$ висит маленький шарик. 
Под этим гвоздём на одной вертикали с ним на расстоянии $l < L$ вбит второй гвоздь. 
Шарик отводят вдоль стены так, что нить принимает горизонтальное положение, и отпускают без толчка. 
Найдите расстояния~$l$, при которых шарик перелетит через нижний гвоздь. Нить невесома и нерастяжима, трения нет.

\AddProb В широкий сосуд с жидкостью частично погружается плоский конденсатор. 
Конденсатор подключен к батарее, которая поддерживает на обкладках конденсатора постоянную разность потенциалов~$U$. 
Расстояние между пластинами~$d$, плотность жидкости~$\rho$, диэлектрическая проницаемость~$\varepsilon$. 
На какую высоту поднимется жидкость в конденсаторе? Поверхностным натяжением пренебречь.

\chapter{Олимпиада "Юные таланты" 2014 года}
\section{10 �����}

% ������ 19, ������ ������������, ��������������� ��������
\begin{wrapfigure}{r}{3.5cm}
\includegraphics[width=3.5cm]{0115RelativityPowerboats.jpg}
\end{wrapfigure}

\AddProb ��� ������ ����� ������������ �� ������� $A$ � $B$, ����������� �� ��������������� ������� ���� � ������������� �� ����������~$L$, 
�������� �� ������~$AB$. ������ $AB$ �������� ���� $\alpha$ � ������������ �������� �������~$v$. �������� �������� ������� ������������ ���� ���������. 
�� ����� ���������� �� ������ $B$ ��������� ������� �������, ���� ��� ����������� ����� ����� $\tau$ ����� ������ �� ��������.

\AddProb ����� ����� $m$ �������� �� ��������� ������� ���������~$k$, ������� ��������� �����~$L_0$, ��� ������� ��������� ������������ ������. 
����� �������� ��������� � ������� ���������~$\omega$. ����� ���� �������� ��� ���� ������� � ����������? 
��������������� ����������� ���� ���������� ������� �� ���������� ������.

\AddProb ��� ��������� ����������� $t$ ������� �����, ��������� �� ������� ����������, ������������ � ��������� � ��� $U$ � ����� ���������� ��������������. 
������������� ������������ ������������� ���������� ������� ����� � ���������� �������������� $\alpha_1$ �~$\alpha_2$, 
� ������������� ���� ���������� ��� ����������� $0^{\circ}C$ ���������. ��� ������� ���������� ����� ������� $1$ � $2$ �� �����������? 
�������, ��� � ��������� ���������� ���������� $\alpha_1t<<1$, $\alpha_2t<<1$.

\AddProb ����� ����������� �� ������� ����~$F$, ����������� �� ��� ��������������� ������� �������~$S$, � ������� �� ������� � ������~$H$ 
�������� ���� ���������~$\rho$. ��������, ��� ������ ������� � ������ �������� ���������� ����� ����, ������~$Q$.



\section{11 �����}

% ������ 19, ������ ������������, ��������������� ��������
\begin{wrapfigure}{r}{3.5cm}
\includegraphics[width=3.5cm]{0115RelativityPowerboats.jpg}
\end{wrapfigure}

\AddProb ��� ������ ����� ������������ �� ������� $A$ � $B$, ����������� �� ��������������� ������� ���� � ������������� �� ����������~$L$, 
�������� �� ������~$AB$. ������ $AB$ �������� ���� $\alpha$ � ������������ �������� �������~$v$. �������� �������� ������� ������������ ���� ���������. 
�� ����� ���������� �� ������ $B$ ��������� ������� �������, ���� ��� ����������� ����� ����� $\tau$ ����� ������ �� ��������.

\AddProb ����� ����� $m$ �������� �� ��������� ������� ���������~$k$, ������� ��������� �����~$L_0$, ��� ������� ��������� ������������ ������. 
����� �������� ��������� � ������� ���������~$\omega$. ����� ���� �������� ��� ���� ������� � ����������? 
��������������� ����������� ���� ���������� ������� �� ���������� ������.

\AddProb ��� ��������� ����������� $t$ ������� �����, ��������� �� ������� ����������, ������������ � ��������� � ��� $U$ � ����� ���������� ��������������. 
������������� ������������ ������������� ���������� ������� ����� � ���������� �������������� $\alpha_1$ �~$\alpha_2$, 
� ������������� ���� ���������� ��� ����������� $0^{\circ}C$ ���������. ��� ������� ���������� ����� ������� $1$ � $2$ �� �����������? 
�������, ��� � ��������� ���������� ���������� $\alpha_1t<<1$, $\alpha_2t<<1$.

\AddProb �� ������� ������� ����������� ������� ��� $\nu$ ����� �����, ��������� �� ���� �������� �������� ����������� �������� $p$ �� ������ $V$ 
� ����� �������. ��������, ��� �� ������� $3 - 1$ ��� ����� ���� ��������� ������ $A$ $(A > 0)$, � ����������� ���� ����������� � $n\,=\,4$ ����. 
����� $2$ � $3$ ����� �� ����� ��������. ����� $1$ � $2$ �� ��������� $pV$ ����� �� ������, ���������� ����� ������ ���������. 
���������� ����������� ���� � ����� $1$ � ������ ���� �� ���� ����.

\end{document}
