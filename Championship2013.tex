\section{I курс}

\AddProb Пассажир первого вагона поезда длины $l$ прогуливался по перрону. Когда он был рядом с последним вагоном, поезд начал двигаться с ускорением $a$. 
Пассажир сразу же побежал со скоростью $v$. Через какое время он догонит вагон?

\AddProb Частица массы $m$ влетает в область, где на нее действует тормозящая сила, пропорциональная расстоянию между частицей и границей области. 
Найдите эту зависимость, если глубина проникновения частицы в область торможения пропорциональна ее начальному импульсу: $l~=~\alpha\,p$.

\AddProb Небольшую шайбу $A$ положили на наклонную плоскость, составляющую угол $\alpha$ с горизонтом, и сообщили начальную скорость $v_0$. 
Найти зависимость скорости шайбы от угла $\varphi$, если коэффициент трения $k~=~ \tg \alpha$ и в начальный момент $\varphi_0~=~\pi/2$.


\section{II курс}

\AddProb Пассажир первого вагона поезда длины $l$ прогуливался по перрону. Когда он был рядом с последним вагоном, поезд начал двигаться с ускорением $a$. 
Пассажир сразу же побежал со скоростью $v$. Через какое время он догонит вагон?

\AddProb Два одинаковых шарика, имеющие одинаковые заряды $q$, соединены пружиной. Шарики колеблются так, 
что расстояние между ними меняется от $l$ до $4l$. Найти жесткость пружины, если ее длина в свободном состоянии равна~$2l$.

\AddProb Замкнутая стеклянная трубка с отводом, погруженным в открытый сверху сосуд с ртутью, в верхней своей части содержит столбик воздуха. 
Его границы с ртутью находятся на расстоянии $R$ от оси симметрии системы. Определите, с какой угловой скоростью нужно вращать систему вокруг этой оси, 
чтобы давление воздуха изменилось в $n$ раз. Начальное давление воздуха $p_0$, плотность ртути $\rho$, ее уровень в сосуде можно считать неизменным.