\section{10 класс}

\AddProb Мимо остановки по прямой улице проезжает грузовик со скоростью 10~м/с. Через 5 с от остановки вдогонку грузовику отъезжает мотоциклист, 
движущийся с ускорением 3~м/$c{^2}$. На каком расстоянии от оснановки мотоциклист догонит грузовик?

\AddProb На гладкий стол помещен брусок массой 1 кг, на котором лежит коробок массой 50~г. Брусок прикреплен к одному из концов невесомой пружины, 
другой конец которой закреплен в неподвижной стенке. Брусок отводят от положения равновесия перпендикулярно стенке на расстояние $L$ 
и отпускают с нулевой начальной скоростью. При каком значении $L$ коробок начнет скользить по бруску? 
Коэффициент трения бруска о брусок 0.2, жесткость пружины~500~Н/м. Трением бруска о стол пренебречь.

\AddProb Шарик массой 100~г подвешен на нити длиной 1~м. Его приводят в движение так, что он вращается по окружности, лежащей в горизонтальной плоскости, 
которая находится на расстоянии 1/2~м от точки подвеса. Какую работу $A$ нужно совершить для реализации такого движения?

\AddProb При движении трамвая по горизонтальному участку пути с некоторой скоростью его двигатель потребляет ток 100 А. КПД двигателя 0.9. 
При движении трамвая по наклонному участку пути вниз с той же скростью двигатель тока не потребляет. Какой ток будет потреблять двигатель при 
движении трамвая по тому же участку пути вверх с той же скоростью? При решении задачи учесть, что КПД двигателя зависит от потребляемого тока.



\section{11 класс}

\AddProb В электрической цепи, показанной на рисунке, ЭДС источника тока равна 12 В, емкость конденсатора $C\,=\,2$~мФ, 
индуктивность катушки $L\,=\,5$~мГн, сопростивление лампы $R\,=\,5$~Ом и сопротивление резистора $r\,=\,3$~Ом. 
Какая энергия выделится в лампе после размыкания ключа? Сопротивлением катушки и проводов пренебречь.

\AddProb Преследуя добычу, гепард движется по прямой горизонтальной тропе прыжками длиной 8 м. Внезапно на пути гепарда встречается овраг глубиной 4/3 м. 
Отталкиваясь от края оврага точно так же, как и при движени по тропе, гепард прыгает в овраг. Найти горизонтальное перемещение гепарда при этом прыжке, 
если горизонтальная составляющая его скорости 108 км/ч. Сопротивление воздуха не учитывать, дно оврага считать горизонтальным.

\AddProb В цилиндре под подвижным поршнем находится идеальный газ, поддерживаемый при постоянной температуре. 
Когда на поршень положили груз массой 1 кг, объем газа уменьшился в 2 раза. Какой массы груз нужно положить на поршень дополнительно, 
чтобы объем газа уменьшился еще в 3 раза?

\AddProb Небольшое тело скользит по вогнутой поверхности, имеющей параболическую форму (уравнение поверхности $y=ax^2$, где а -- константа). 
Коэффициент трения между телом и поверхностью $\mu \ll 1$. В начальный момент времени координата тела $x_0\,=\,2.75\,\mu/a$, а скорость нулевая. 
Какова будет координата тела, когда оно окончательно остановится?