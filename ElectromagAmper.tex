\section{Сила Ампера}

\begin{ex}
\hspace{0pt} \\
\begin{minipage}{.65\textwidth}
(2016) На рисунке представлена модель электродвигателя. Замкнутый контур образован двумя вертикальными рейками, между концами которых включен источник постоянного тока с ЭДС $\mathcal{E}$, a другие концы замкнуты перемычкой сопротивлением $R$ и длиной $L$. Перемычка за счет скользящих контактов может без трения скользить вдоль реек. Контур находится в однородном магнитном поле с индукцией $B$, направленной горизонтально. Известно, что если к перемычке подвесить груз массы $M$, она будет в состоянии равновесия. 1) Определите массу $m$ перемычки. 2) Определите установившуюся скорость ненагруженной перемычки. Сопротивлением реек и внутренним сопротивлением источника пренебречь.
\end{minipage}
\begin{minipage}{.35\textwidth}
\centering
\includestandalone{Pictures/RodInMagneticField}
\end{minipage}
\begin{ans}
1) $m=\mathcal{E}BL/Rg - M$; 2) $v = \frac{MgR}{B^2L^2}$
\end{ans}
\end{ex}

%Черепанов
\begin{ex}
(2003) Металлический стержень массой $m$ и длиной $L$ подвешен на двух легких проводах длиной $l$ в магнитном поле с индукцией $B$, 
вектор которой направлен вертикально. К точкам крепления проводов подключен конденсатор емкостью $C$, заряженный до напряжения $U$. 
Сопротивление стержня и проводов пренебрежимо мало. Найти максимальный угол отклонения проводов от вертикали, если разрядка конденсатора происходит за очень малое время.
\begin{ans}
$\sin \frac {\alpha}{2} = \frac{CUlB}{2m \sqrt{gL}}$
\end{ans}
\end{ex}

\begin{ex}
(2010) Посередине между двумя жестко закрепленными проводниками с током на расстоянии $a$ расположен груз массы $m$, представляющий собой цилиндрическую железную трубку длиной $l_1$ и укрепленный с помощью упругих растяжек длиной $l$. 
Магнитная проницаемость железа $\mu$. Внутри растяжек установлен еще один проводник. По всем трем проводникам течет ток $I$. 
Определите собственную частоту свободных колебаний груза, считая, что в процессе колебаний натяжение растяжек $T_0$ не изменяется. 
Определить зависимость критического значения силы тока от натяжения растяжек, считая критическим значением такое значение, при котором колебания в системе невозможны.
\begin{center}
\includestandalone{Pictures/122010AmperesForceLawTwoConductors}
\end{center}
\begin{ans}
$\omega^2 = \frac{2 T_0}{ml} - \frac{\mu \mu_0 I l_1}{\pi a^2}$, $I^* = \frac{2T_0 \pi a^2}{\mu \mu_0 l l_1}$
\end{ans}
\end{ex}

\begin{ex}
(2017) Одна из моделей генератора постоянного тока представляет собой проводящий диск, который вращается в однородном магнитном поле с индукцией,
направленной перпендикулярно плоскости вращения диска. Если концы некоторого проводника сопротивлением $R$ присоединить к центру диска и через скользящий контакт к его ободу, то в цепи возникнет электрический ток. 1) Объясните возникновение электрического тока и найдите силу тока, если радиус диска $r = 10$ см, частота вращения диска $\nu = 40$ об/с, индукция магнитного поля $B = 0,1$ Тл, сопротивление нагрузки $R = 0,5$ Ом. 2) Какая мощность затрачивается для поддержания вращения диска? 3) Какой момент силы относительно оси вращения нужно прикладывать к диску? Сопротивлениями диска и контактов пренебречь.
\begin{center}
\includestandalone{Pictures/AmperGenerator}
\end{center}
\begin{ans}
1) $I=\pi v B r^2/R = 251$ мА; 2) $P = I^2 R = 31,6$ мВт; 3) $M = P/2 \pi \nu = 126$ мкН м
\end{ans}
\end{ex}

\begin{ex}
\hspace{0pt} \\
\begin{minipage}{.65\textwidth}
(2009) Медная монета массой $m$ радиусом $R$ и толщиной $d$ движется в поле силы тяжести в однородном магнитном поле $B$. 
Вектор индукции магнитного поля направлен вдоль оси монеты и перпендикулярно ускорению свободного падения. Найти ускорение монеты.
\end{minipage}
\begin{minipage}{.35\textwidth}
\centering
\includestandalone[scale=0.9]{Pictures/122009EMICoin}
\end{minipage}
\begin{ans}
$a=g/(1+ \varepsilon_0 \pi R^2 d B^2 / m)$
\end{ans}
\end{ex}

\section{Закон Био-Савара-Лапласа}

%Иродов2.234
\begin{ex}
\hspace{0pt} \\
\begin{minipage}{.65\textwidth}
(1998) Ток $I$ течет по длинному прямому проводнику, сечение которого имеет форму тонкого полукольца радиуса $R$. Найти магнитную индукцию на оси~$O$.
\end{minipage}
\begin{minipage}{.35\textwidth}
\centering
\includestandalone{Pictures/HalfringMagn}
\end{minipage}
\begin{ans}
$B = \frac{\mu_0 I}{\pi^2 R}$
\end{ans}
\end{ex}

%Иродов2.265б
\begin{ex}
\hspace{0pt} \\
\begin{minipage}{.65\textwidth}
Найти модуль и направление силы, действующей на единицу длины тонкого проводника с током $I$ в точке $O$, если проводник изогнут так, как показано на рисунке.
\end{minipage}
\begin{minipage}{.35\textwidth}
\centering
\includestandalone{Pictures/1202BiotSavartLawConductor}
\end{minipage}
\begin{ans}
$F = \frac{\mu_0 I^2}{\pi l}$
\end{ans}
\end{ex}

\section{Теорема о циркуляции индукции магнитного поля}

%Ижевск
\begin{ex}
\hspace{0pt} \\
\begin{minipage}{.65\textwidth}
Однородный диэлектрический диск массой $m$ радиуса $R$, равномерно заряженный с полным зарядом $q$, помещен в однородное магнитное поле с индукцией $B$. Какую угловую скорость получит диск, если выключить магнитное поле?
\end{minipage}
\begin{minipage}{.35\textwidth}
\centering
\includestandalone[width = 0.8 \textwidth]{Pictures/1205AmperesCircuitalLawDisc}
\end{minipage}
\begin{ans}
$\omega = qB/2m$
\end{ans}
\end{ex}

%Ижевск
\begin{ex}
По поверхности жесткой непроводящей однородной сферы массой $m$ равномерно распределен заряд $q$. 
Сфера может свободно вращаться вокруг своей вертикальной оси. В начальный момент сфера покоилась, а магнитное поле было равно нулю. 
Найти, как меняется со временем угловая скорость сферы при включении однородного магнитного поля, 
сонаправленного с осью вращения сферы и меняющегося во времени по заданному закону $B(t)$.
\begin{ans}
$\omega(t) = qB(t)/2m$
\end{ans}
\end{ex}