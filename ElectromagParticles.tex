\section{Движение заряженных частиц}

\begin{ex}
Заряженная частица массы $m$ влетает в магнитное поле $B$ под углом $\alpha$ со скоростью $v$. По какой траектории движется частица? Каков пространственный период витка (шаг спирали)?
\begin{ans}
\end{ans}
\end{ex}

%Зональные студенческие олимпиады Ижевска
\begin{ex}
По обмотке длинного цилиндрического соленоида радиуса $R$ протекает постоянный ток, создающий внутри соленоида однородное магнитное поле с индукцией $B$. Между витками соленоида в него влетает по радиусу (перпендикулярно оси соленоида) электрон со скоростью $v$. Отклоняясь в магнитном поле, электрон спустя некоторое время покинул соленоид. Определите время движения внутри соленоида.
\begin{ans}
\end{ans}
\end{ex}

\begin{ex}
(2008) Электронно-лучевая трубка помещена в однородное магнитное поле, напряженность $H$ которого перпендикулярна плоскости экрана. Электроны влетают в электронно-лучевую трубку из электронной пушки с составляющей скорости $u$ вдоль оси трубки и составляющей скорости $v_0$ перпендикулярно оси. При какой длине $L$ трубки все электроны фокусируются в одной точке экрана?
\begin{ans}
\end{ans}
\end{ex}

\begin{ex}
(2012) Две заряженные частицы движутся в однородном магнитном поле $B$, причем $q_1/m_1 = q_2/m_2$. Написать уравнения движения центра масс и уравнение относительного движения.
\begin{ans}
\end{ans}
\end{ex}

\begin{ex}
(2013) Заряд $q$ движется в поле магнитного монополя $\vec{B} = \alpha \vec{r}/r^3$. Найдите интеграл движения, следующий из закона изменения момента импульса заряда.
\begin{ans}
\end{ans}
\end{ex}

\begin{ex}
Частица c зарядом $q$ и массой $m$ движется с начальной скоростью $v_0$ в вязкой среде в поперечном магнитном поле с индукцией $B$. Сила сопротивления $\vec{F} = -\gamma \vec{v}$, где $\gamma$ - константа. На каком расстоянии от начальной точки частица остановится?
\begin{ans}
\end{ans}
\end{ex}

\begin{ex}
(2007) Частица c зарядом $q$ и массой $m$ движется в постоянных однородных скрещенных полях $\vec{E} \bot \vec{H}$ в среде с малым линейным сопротивлением $\vec{F} = -\gamma \vec{v}$. Найти скорость частицы вдоль поля $\vec{E}$, усредненную по периоду.
\begin{ans}
\end{ans}
\end{ex}

%Зональные студенческие олимпиады Ижевска, Черепанов
\begin{ex}
На магнитный барьер, задаваемый в пространстве статическим магнитным полем $\vec{B} = \left(0, 0, \frac{B_0}{\cosh^2(ky)}\right)$, где $k$ -- константа, из бесконечности налетает протон с начальной скоростью $\vec{v}_{-\infty} = (0, v_0, 0)$, $\vec{r}_{-\infty} = (0, -\infty, 0)$. Оцените минимальную скорость, которую должен иметь протон, чтобы преодолеть барьер и уйти на бесконечность. 
\begin{ans}
\end{ans}
\end{ex}

\begin{ex}
\hspace{0pt} \\
\begin{minipage}{.65\textwidth}
(2018) Вакуумный диод представляет собой две металлические пластины -- катод и анод. Между пластинами имеется однородное магнитное поле с индукцией $B$, параллельной плоскости пластин (направленной из плоскости чертежа). Расстояние между пластинами $d$. Из катода вылетают электроны. 1) При каких начальных скоростях все электроны не смогут достичь анода при $U = 0$? 2) При каких напряжениях $U$ все электроны не смогут достичь анода при нулевой начальной скорости?
\end{minipage}
\begin{minipage}{.35\textwidth}
\centering
\includegraphics[width = 0.9 \textwidth]{ElectronInVacuumDiode.png}
\end{minipage}
\begin{ans}
\end{ans}
\end{ex}

%Черепанов
\begin{ex}
(2005)  Магнетрон -- это прибор, состоящий из нити накала радиуса $a$ и коаксиального цилиндрического анода радиуса $b$, которые находятся в однородном магнитном поле параллельном нити. Между нитью и анодом приложена ускоряющая разность потенциалов $U$. Найти минимальное значение индукции магнитного поля $B$, при котором электроны, вылетающие с нулевой начальной скоростью из нити, не будут достигать анода.
\begin{ans}
\end{ans}
\end{ex}

\begin{ex}
(2014) Незаряженная неподвижная частица распалась в однородном магнитном поле с индукцией $B$ на две частицы с массами $m_1$ и $m_2$ и зарядами $+q$ и $-q$. Найдите время, через которое произойдет соударение частиц. Кулоновским взаимодействием между частицами пренебречь.
\begin{ans}
\end{ans}
\end{ex}